% Options for packages loaded elsewhere
\PassOptionsToPackage{unicode}{hyperref}
\PassOptionsToPackage{hyphens}{url}
%
\documentclass[
  a4paper,
]{article}
\usepackage{amsmath,amssymb}
\usepackage{setspace}
\usepackage{iftex}
\ifPDFTeX
  \usepackage[T1]{fontenc}
  \usepackage[utf8]{inputenc}
  \usepackage{textcomp} % provide euro and other symbols
\else % if luatex or xetex
  \usepackage{unicode-math} % this also loads fontspec
  \defaultfontfeatures{Scale=MatchLowercase}
  \defaultfontfeatures[\rmfamily]{Ligatures=TeX,Scale=1}
\fi
\usepackage{lmodern}
\ifPDFTeX\else
  % xetex/luatex font selection
\fi
% Use upquote if available, for straight quotes in verbatim environments
\IfFileExists{upquote.sty}{\usepackage{upquote}}{}
\IfFileExists{microtype.sty}{% use microtype if available
  \usepackage[]{microtype}
  \UseMicrotypeSet[protrusion]{basicmath} % disable protrusion for tt fonts
}{}
\makeatletter
\@ifundefined{KOMAClassName}{% if non-KOMA class
  \IfFileExists{parskip.sty}{%
    \usepackage{parskip}
  }{% else
    \setlength{\parindent}{0pt}
    \setlength{\parskip}{6pt plus 2pt minus 1pt}}
}{% if KOMA class
  \KOMAoptions{parskip=half}}
\makeatother
\usepackage{xcolor}
\usepackage[margin=1in]{geometry}
\usepackage{graphicx}
\makeatletter
\def\maxwidth{\ifdim\Gin@nat@width>\linewidth\linewidth\else\Gin@nat@width\fi}
\def\maxheight{\ifdim\Gin@nat@height>\textheight\textheight\else\Gin@nat@height\fi}
\makeatother
% Scale images if necessary, so that they will not overflow the page
% margins by default, and it is still possible to overwrite the defaults
% using explicit options in \includegraphics[width, height, ...]{}
\setkeys{Gin}{width=\maxwidth,height=\maxheight,keepaspectratio}
% Set default figure placement to htbp
\makeatletter
\def\fps@figure{htbp}
\makeatother
\setlength{\emergencystretch}{3em} % prevent overfull lines
\providecommand{\tightlist}{%
  \setlength{\itemsep}{0pt}\setlength{\parskip}{0pt}}
\setcounter{secnumdepth}{-\maxdimen} % remove section numbering
\ifLuaTeX
\usepackage[bidi=basic]{babel}
\else
\usepackage[bidi=default]{babel}
\fi
\babelprovide[main,import]{catalan}
% get rid of language-specific shorthands (see #6817):
\let\LanguageShortHands\languageshorthands
\def\languageshorthands#1{}
\ifLuaTeX
  \usepackage{selnolig}  % disable illegal ligatures
\fi
\usepackage{bookmark}
\IfFileExists{xurl.sty}{\usepackage{xurl}}{} % add URL line breaks if available
\urlstyle{same}
\hypersetup{
  pdftitle={U2. Windows Server. Instal·lació i ús (II)},
  pdfauthor={@tofermos 2024},
  pdflang={ca-ES},
  hidelinks,
  pdfcreator={LaTeX via pandoc}}

\title{U2. Windows Server. Instal·lació i ús (II)}
\author{@tofermos 2024}
\date{}

\begin{document}
\maketitle

{
\setcounter{tocdepth}{2}
\tableofcontents
}
\setstretch{1.5}
\newpage
\renewcommand\tablename{Tabla}

\section{1 INSTAL·LACIÓ DEL WINDOWS SERVER 2019 AMB ENTORN
GRÀFIC}\label{installaciuxf3-del-windows-server-2019-amb-entorn-gruxe0fic}

\textbf{Resum}

Un primer pas per emular una Domini serà la instal·lació d'un Windows
Server i dos màquines de Windows 1X.

\begin{enumerate}
\def\labelenumi{\arabic{enumi}.}
\item
  Configurarem les MV com a Xarxa Interna. Emulem una xarxa local de
  computadores connectades a un switch.
\item
  Configuració mitjançant IP fixes. IPs privades en la mateixa xarxa.
\item
  Assegurarem la connectivitat entre màquines. (Detecció de xarxes i
  compartició en Windows, Firewall\ldots)
\end{enumerate}

\section{2 CONFIGURAR EL SERVIDOR}\label{configurar-el-servidor}

\section{2.1 ``Xarxa Interna'' en
VirtualBox.}\label{xarxa-interna-en-virtualbox.}

Afegirem 2 targes, una per a emular la xarxa local ( Xarxa interna ) i
l'altra per disposar de la connexió d'Internet de l'amfitrió.

En el WINDOWS 1x hem de tindre NOMÉS la tarja interna.

\section{2.2 Configuració de la xarxa local en
Windows}\label{configuraciuxf3-de-la-xarxa-local-en-windows}

\emph{Windows+R: Configuración, Red e internet, Centro de Redes y
Recursos Compartidos}

Ho hem de revisar en \textbf{LES DOS MÀQUINES}

\textbf{Detecció de xarxa i ús compartit}

\emph{Win+R: Configuración, Red e internet, Centro de Redes y Recursos
Compaçartidos, Cambiar configuración del Uso compartido avanzado:}

\begin{itemize}
\tightlist
\item
  Activar la detección de redes
\item
  Activar el uso compartido de carpetas e impresoras.
\end{itemize}

Ho hem de revisar en \textbf{LES DOS MÀQUINES}

\textbf{IPs privades en la mateixa xarxa.}

IP Windows 1X: 192.168.0.2/24\\
IP Windows Server: 192.168.0.1/24

\emph{Windows+R: Configuración, Red e internet, Centro de Redes y
Recursos Compartidos, Ethernet}

\subsection{2.3 Provar la connectivitat amb el protocol ICMP
(ping)}\label{provar-la-connectivitat-amb-el-protocol-icmp-ping}

\textbf{Revisar les les restriccions del FireWall de Windows}

Ho hem de revisar en \textbf{LES DOS MÀQUINES}

\textbf{Provar el ICMP}

\subsection{2.4 Canviar el nom del
servidor}\label{canviar-el-nom-del-servidor}

\subsubsection{Revisar aspectes bàsics de la
configuració}\label{revisar-aspectes-buxe0sics-de-la-configuraciuxf3}

Un exemple podria ser desactivar/activar el Servei d'actualitzacions.

Altre exemple podria ser assegurar la Zona horària.

\end{document}
