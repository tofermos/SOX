% Options for packages loaded elsewhere
\PassOptionsToPackage{unicode}{hyperref}
\PassOptionsToPackage{hyphens}{url}
%
\documentclass[
  12 pt,
  a4paper,
]{article}
\usepackage{amsmath,amssymb}
\usepackage{setspace}
\usepackage{iftex}
\ifPDFTeX
  \usepackage[T1]{fontenc}
  \usepackage[utf8]{inputenc}
  \usepackage{textcomp} % provide euro and other symbols
\else % if luatex or xetex
  \usepackage{unicode-math} % this also loads fontspec
  \defaultfontfeatures{Scale=MatchLowercase}
  \defaultfontfeatures[\rmfamily]{Ligatures=TeX,Scale=1}
\fi
\usepackage{lmodern}
\ifPDFTeX\else
  % xetex/luatex font selection
  \setmainfont[]{Times New Roman}
\fi
% Use upquote if available, for straight quotes in verbatim environments
\IfFileExists{upquote.sty}{\usepackage{upquote}}{}
\IfFileExists{microtype.sty}{% use microtype if available
  \usepackage[]{microtype}
  \UseMicrotypeSet[protrusion]{basicmath} % disable protrusion for tt fonts
}{}
\makeatletter
\@ifundefined{KOMAClassName}{% if non-KOMA class
  \IfFileExists{parskip.sty}{%
    \usepackage{parskip}
  }{% else
    \setlength{\parindent}{0pt}
    \setlength{\parskip}{6pt plus 2pt minus 1pt}}
}{% if KOMA class
  \KOMAoptions{parskip=half}}
\makeatother
\usepackage{xcolor}
\usepackage[margin=1in]{geometry}
\usepackage{graphicx}
\makeatletter
\def\maxwidth{\ifdim\Gin@nat@width>\linewidth\linewidth\else\Gin@nat@width\fi}
\def\maxheight{\ifdim\Gin@nat@height>\textheight\textheight\else\Gin@nat@height\fi}
\makeatother
% Scale images if necessary, so that they will not overflow the page
% margins by default, and it is still possible to overwrite the defaults
% using explicit options in \includegraphics[width, height, ...]{}
\setkeys{Gin}{width=\maxwidth,height=\maxheight,keepaspectratio}
% Set default figure placement to htbp
\makeatletter
\def\fps@figure{htbp}
\makeatother
\setlength{\emergencystretch}{3em} % prevent overfull lines
\providecommand{\tightlist}{%
  \setlength{\itemsep}{0pt}\setlength{\parskip}{0pt}}
\setcounter{secnumdepth}{-\maxdimen} % remove section numbering
\ifLuaTeX
\usepackage[bidi=basic]{babel}
\else
\usepackage[bidi=default]{babel}
\fi
\babelprovide[main,import]{spanish}
\ifPDFTeX
\else
\babelfont{rm}[]{Times New Roman}
\fi
% get rid of language-specific shorthands (see #6817):
\let\LanguageShortHands\languageshorthands
\def\languageshorthands#1{}
\ifLuaTeX
  \usepackage{selnolig}  % disable illegal ligatures
\fi
\usepackage{bookmark}
\IfFileExists{xurl.sty}{\usepackage{xurl}}{} % add URL line breaks if available
\urlstyle{same}
\hypersetup{
  pdfauthor={Tomàs Ferrandis Moscardó},
  pdflang={es-ES},
  hidelinks,
  pdfcreator={LaTeX via pandoc}}

\title{U7-OpenLDAP}
\usepackage{etoolbox}
\makeatletter
\providecommand{\subtitle}[1]{% add subtitle to \maketitle
  \apptocmd{\@title}{\par {\large #1 \par}}{}{}
}
\makeatother
\subtitle{Introducció i instal·lació}
\author{Tomàs Ferrandis Moscardó}
\date{}

\begin{document}
\maketitle

\setstretch{1.5}
\begin{verbatim}
 1  LDAP. Introducció
\end{verbatim}

LDAP significa Lightweight Directory Access Protocol. Com el seu nom
indica, és un protocol lleuger en mode client-servidor per accedir als
serveis de directori, específicament basats en els serveis de directori
X.500. S'executa sobre TCP/IP o altres protocols orientats a connexió.
LDAP es defineix a l'estàndard RFC2251. S'utilitza comunament per a
emmagatzemar informació sobre organitzacions, usuaris, xarxes, etc.

Un directori (no confondre amb un directori del nostre disc dur, ja que
és una estructura molt més àmplia) és similar a una base de dades, però
tendeix a contenir més informació descriptiva, basada en atributs
(recordem els atributs típics d'un arxiu en un directori local: només
lectura, invisible, data de creació, etc\ldots). En un directori,
normalment, la informació es llegeix més que no pas s'escriu. Els
serveis de directori habitualment estan optimitzats per a donar una
ràpida resposta en operacions de cerca o exploració. També poden tenir
la capacitat de replicar (en diversos servidors físics) la informació
continguda en un directori a fi i efecte de millorar la disponibilitat
de les dades i la fiabilitat. Com que la replicació de dades pot generar
inconsistències, temporalment es sincronitzen les dades per a evitar-ho.

Hi ha moltes maneres diferents de proporcionar un servei de directori.
Els diferents mètodes permeten que diferents tipus d'informació
s'emmagatzemen en el directori, establir requisits diferents per a la
forma en què la informació es pot referenciar, consultar i actualitzar,
la manera com està protegida d'accessos no autoritzats, etc. Alguns
serveis de directori són locals, proporcionant serveis a un context
restringit (per exemple, el servei de finger en una única màquina).
Altres serveis són globals, proporcionant serveis a un context molt més
ampli.

\begin{verbatim}
     1.1  Com funciona LDAP?
\end{verbatim}

El funcionament, com hem dit abans, està basat en un model
client-servidor. Un client LDAP es connecta a un servidor LDAP i li fa
una consulta. El servidor contesta amb la resposta, o amb un apuntador
on el client pot obtenir més informació (típicament un altre servidor
LDAP). Dèiem abans que poden haver molts servidors amb les dades
replicades: per tant no és problema que un client es connecti amb un
servidor o a un altre; el client veurà sempre la mateixa vista del
directori. Aquesta és una característica molt important d'un servei
global de directori com LDAP.

\begin{verbatim}
     1.2  Avantatges en l'ús de LDAP
\end{verbatim}

Un directori LDAP destaca sobre els altres tipus de bases de dades per
les següents característiques:

\begin{verbatim}
• És molt ràpid en la lectura de registres.
• Permet  replicar el servidor de forma molt senzilla i econòmica.
• Moltes aplicacions de tot tipus tenen interfícies de connexió a LDAP i es poden integrar fàcilment.
• Disposa d'un model de noms globals que assegura que totes les entrades són úniques.
• Utilitza un sistema jeràrquic d'emmagatzematge d'informació.
• Permet múltiples directoris independents
• Funciona sobre TCP/IP i SSL
• La majoria de servidors LDAP són fàcils d'instal·lar, mantenir i optimitzar.  
         1.2.1  Usos pràctics de LDAP
\end{verbatim}

Donades les característiques de LDAP seus usos més comuns són:

\begin{verbatim}
• Directoris d'informació. Per exemple bases de dades d'empleats organitzats per departaments (seguint l'estructura organitzativa de l'empresa) o qualsevol tipus de pàgines grogues.
• Sistemes d'autenticació / autorització centralitzada. Grans sistemes on es guarda gran quantitat de registres i es requereix un ús constant dels mateixos. Per exemple: Active Directory Server de Microsoft, per gestionar tots els comptes d'accés a una xarxa corporativa i mantenir centralitzada la gestió de l'accés als recursos.
• Sistemes d'autenticació per a pàgines web, alguns dels gestors de continguts més coneguts disposen de sistemes d'autenticació a través de LDAP.   
• Sistemes de control d'entrades a edificis, oficines ....
• Sistemes de correu electrònic. Grans sistemes formats per més d'un servidor que accedeixin a un repositori de dades comú.
• Sistemes d'allotjament de pàgines web i FTP, amb el repositori de dades d'usuari compartit.
• Grans sistemes d'autenticació basats en RADIUS, per al control d'accessos dels usuaris a una xarxa de connexió o ISP.  
• Servidors de certificats públics i claus de seguretat.
• Autenticació única o "single sign-on" per a la personalització d'aplicacions.
• Perfils d'usuaris centralitzats, per permetre itinerància o "Roaming"
• Llibretes d'adreces compartides.
\end{verbatim}

Alguns exemples

Sistema de correu electrònic Cada usuari s'identifica per la seva adreça
de correu electrònic, els atributs que es guarden de cada usuari són la
seva contrasenya, el seu límit d'emmagatzematge (quota), la ruta del
disc dur on s'emmagatzemen els missatges (bústia) i possiblement
atributs addicionals per activar sistemes anti-spam o antivirus.

Com es pot veure aquest sistema LDAP rebrà centenars de consultes cada
dia (una per cada correu electrònic rebut i una cada vegada que l'usuari
es connecta mitjançant POP3 o webmail). No obstant el nombre de
modificacions diàries és molt baix, ja que només es pot canviar la
contrasenya o donar de baixa a l'usuari, operacions ambdues que no es
realitzen de forma freqüent.

Sistema d'autenticació a una xarxa Cada usuari s'identifica per un nom
d'usuari i els atributs assignats són la contrasenya, els permisos
d'accés, els grups de treball als quals pertany, la data de caducitat de
la contrasenya, etc\ldots{}

Aquest sistema rebrà una consulta cada vegada que l'usuari accedeixi a
la xarxa i una més cada vegada que accedeixi als recursos del grup de
treball (directoris compartits, impressores \ldots) per comprovar els
permisos de l'usuari. Enfront d'aquests centenars de consultes només
unes poques vegades es canvia la contrasenya d'un usuari o se l'inclou
en un nou grup de treball. 2 Estructura d'una base de dades/directori
LDAP 2.1 Entrades, objectes i atributs Com hem dit abans, una base de
dades LDAP té una estructura jeràrquica. Bàsicament totes les dades
s'emmagatzemen en alguna part del directori LDAP, i a similitud dels
directoris de fitxers, aquest directori s'organitza en arbre.

Veiem primer, el punt i final del directori, que és l'entrada o objecte.
El model d'informació de LDAP està basat en entrades. Una entrada és una
col·lecció d'atributs que tenen un Nom Distintiu o Distinguished Name
(identificat com DN) únic i global. El DN s'utilitza per referir-se a
una entrada sense ambigüitats. Cada atribut d'una entrada té un tipus i
un o més valors i son els que contenen la informació associada a
l'objecte. Els tipus són normalment paraules mnemotècniques, com ``cn''
per common name, o ``mail'' per una adreça de correu.

En comparació amb una base de dades relacional, una entrada seria com un
registre. L'atribut seria el camp.

\begin{verbatim}
     2.2  Estructura de l’atribut DN i una breu introducció històrica
         2.2.1  Introducció històrica
\end{verbatim}

El nivell superior d'un directori LDAP és la base, conegut com el ``DN
base''. Un DN base, generalment, pren una de les tres formes llistades
ací. Suposem que treballes o estudies a l'institut Maria Enriquez de
Gandia, el qual està a Internet a iesmariaenriquez.es.

o = ``IES Maria Enriquez'', c = ES (DN base en format X.500)

En aquest exemple, o = IES Maria Enriquez es refereix a l'organització,
que en aquest context hauria de ser tractada com un sinònim del nom de
l'empresa. c = ES indica que la localització general de l'empresa està a
ES. Hi havia una vegada en què aquest va ser el mètode d'especificar la
teva DN base. Els temps i les modes canvien, però, aquests dies, la
majoria de les empreses estan (o planegen estar) a Internet. I amb la
globalització d'Internet, utilitzar un codi de país a la base DN
probablement faça les coses més confuses al final. Amb el temps, el
format X.500 ha evolucionat a altres formats llistats més avall.

o = iesmariaenriquez.es (DN base derivat de la presència a Internet de
l'empresa)

Aquest format és bastant senzill, utilitzant el nom de domini de
l'empresa com a base. Una vegada has passat la porció o = (la qual ve de
organization =), qualsevol a la teva empresa hauria de saber d'on ve la
resta. Aquest va ser, fins fa poc, probablement el més comú dels formats
usats actualment.

dc = iesmariaenriquez, dc = ES (DN base derivat dels components de
domini DNS de l'empresa)

Com el format previ, aquest utilitza el nom de domini DNS com la seva
base. Però on l'altre format deixa el nom de domini intacte (i així
llegible per les persones), aquest format està separat en components de
domini: iesmariaenriquez.es esdevé dc = iesmariaenriquez, dc = es. En
teoria, això pot ser lleument més versàtil, encara que és una mica més
dur de recordar per als usuaris finals.

Aquest és el format recomanable per a noves instal·lacions. Si estàs
planejant utilitzar Active Directory, Microsoft ja ha decidit per tu que
aquest és el format que necessites .

\begin{verbatim}
         2.2.2  Com organitzar les teues dades en el teu arbre de directori
\end{verbatim}

En un sistema de fitxers UNIX, el nivell més alt és l'arrel (/). Per
sota de l'arrel tens molts fitxers i directoris. Com es comentava
anteriorment els directoris LDAP estan configurats en gran part de la
mateixa manera.

Sota la teva base de directori, voldràs crear contenidors que separin
lògicament les teves dades. Per raons històriques (X.500), la majoria
dels directoris configuren aquestes separacions lògiques com a entrades
OU. OU ve de ``Unitats organitzacionals'' (Organizational Units, en
anglès), que en X.500 eren utilitzades per indicar l'organització
funcional dins de l'empresa: vendes, finances, etc. Actualment les
implementacions de LDAP han mantingut la convenció del nom ou =, però
separa les coses per categories àmplies com ou = gent (ou = people), ou
= grups (ou = groups), ou = dispositius (ou = devices), i altres.

\begin{verbatim}
         2.2.3  El DN d'una entrada LDAP
\end{verbatim}

Totes les entrades emmagatzemades en un directori LDAP tenen un únic
``Distinguished Name,'' o DN. El DN per a cada entrada està compost de
dos parts: el Nom Relatiu Distingit (RDN per les seves sigles en anglès,
Relative Distinguished Name) i la localització dins del directori LDAP
on el registre resideix.

El RDN és la porció de la teva DN que no està relacionada amb
l'estructura de l'arbre de directori. La majoria dels ítems que
emmagatzemes en un directori LDAP tindrà un nom, i el nom és
emmagatzemat freqüentment en l'atribut cn (Common Name). Ja que
pràcticament tot té un nom, la majoria dels objectes que emmagatzemarà
LDAP utilitzen el seu valor cn com a base per a la seva RDN. Si estic
emmagatzemant un registre per la meva recepta preferida de menjar de
civada, estaré utilitzant cn=MenjardeCivadaDeluxe com el RDN de la meva
entrada.

\begin{verbatim}
• El DN base del meu directori és dc=iesmariaenriquez, dc=es
• El RDN d’un registre d’un grup cn=alumnes
\end{verbatim}

Atès tot això, quin és el DN complet del registre LDAP per a aquesta
grup? Recorda, es llegeix en ordre invers, cap a enrere - com els noms
de màquina en els DNS.

cn = alumnes, ou = groups, dc = iesmariaenriquez, dc = es

Ara és el moment d'abordar el DN d'un membre del nostre institut. Per
als comptes d'usuari, típicament veuràs un DN basat en el cn o al uid
(ID de l'usuari). Per exemple, el DN del professor Armand Mata (nom de
login: armandmata) pot semblar-se a un d'aquests dos formats:

uid = armandmata, ou = professorat, ou=people, dc = iesmariaenriquez, dc
= es (basat en el login)

LDAP (i X.500) utilitzen uid per a indicar ``ID de l'usuari'', no s'ha
de confondre amb el número uid de UNIX. La majoria de les empreses
intenten donar a cadascun un nom de login, així aquesta aproximació fa
que tinga sentit emmagatzemar informació sobre els empleats. No t'has de
preocupar sobre què faràs quan entre un nou professor amb el mateix nom,
o si el mateix professor decideix canviar-se el nom. No has de canviar
el DN de l'entrada LDAP.

cn = ArmandMata, ou = professorat, ou=people, dc = iesmariaenriquez, dc
= es (basat en el nom)

Aquí veiem l'entrada Nom Comú o CN (per les seves sigles en anglès,
common name) utilitzada. En el cas d'un registre LDAP per a una persona,
pensa en el nom comú com els seu nom complet. Un pot veure fàcilment
l'efecte col·lateral d'aquesta forma: si el nom canvia, el registre LDAP
ha de ``moure'' d'un DN a un altre. Com s'indica anteriorment, has
d'evitar canviar en DN d'una entrada sempre que siga possible.

\end{document}
