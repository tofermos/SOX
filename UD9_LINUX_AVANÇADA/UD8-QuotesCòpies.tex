% Options for packages loaded elsewhere
\PassOptionsToPackage{unicode}{hyperref}
\PassOptionsToPackage{hyphens}{url}
%
\documentclass[
  12 pt,
  a4paper,
]{article}
\usepackage{amsmath,amssymb}
\usepackage{setspace}
\usepackage{iftex}
\ifPDFTeX
  \usepackage[T1]{fontenc}
  \usepackage[utf8]{inputenc}
  \usepackage{textcomp} % provide euro and other symbols
\else % if luatex or xetex
  \usepackage{unicode-math} % this also loads fontspec
  \defaultfontfeatures{Scale=MatchLowercase}
  \defaultfontfeatures[\rmfamily]{Ligatures=TeX,Scale=1}
\fi
\usepackage{lmodern}
\ifPDFTeX\else
  % xetex/luatex font selection
  \setmainfont[]{Times New Roman}
\fi
% Use upquote if available, for straight quotes in verbatim environments
\IfFileExists{upquote.sty}{\usepackage{upquote}}{}
\IfFileExists{microtype.sty}{% use microtype if available
  \usepackage[]{microtype}
  \UseMicrotypeSet[protrusion]{basicmath} % disable protrusion for tt fonts
}{}
\makeatletter
\@ifundefined{KOMAClassName}{% if non-KOMA class
  \IfFileExists{parskip.sty}{%
    \usepackage{parskip}
  }{% else
    \setlength{\parindent}{0pt}
    \setlength{\parskip}{6pt plus 2pt minus 1pt}}
}{% if KOMA class
  \KOMAoptions{parskip=half}}
\makeatother
\usepackage{xcolor}
\usepackage[margin=1in]{geometry}
\usepackage{color}
\usepackage{fancyvrb}
\newcommand{\VerbBar}{|}
\newcommand{\VERB}{\Verb[commandchars=\\\{\}]}
\DefineVerbatimEnvironment{Highlighting}{Verbatim}{commandchars=\\\{\}}
% Add ',fontsize=\small' for more characters per line
\usepackage{framed}
\definecolor{shadecolor}{RGB}{248,248,248}
\newenvironment{Shaded}{\begin{snugshade}}{\end{snugshade}}
\newcommand{\AlertTok}[1]{\textcolor[rgb]{0.94,0.16,0.16}{#1}}
\newcommand{\AnnotationTok}[1]{\textcolor[rgb]{0.56,0.35,0.01}{\textbf{\textit{#1}}}}
\newcommand{\AttributeTok}[1]{\textcolor[rgb]{0.13,0.29,0.53}{#1}}
\newcommand{\BaseNTok}[1]{\textcolor[rgb]{0.00,0.00,0.81}{#1}}
\newcommand{\BuiltInTok}[1]{#1}
\newcommand{\CharTok}[1]{\textcolor[rgb]{0.31,0.60,0.02}{#1}}
\newcommand{\CommentTok}[1]{\textcolor[rgb]{0.56,0.35,0.01}{\textit{#1}}}
\newcommand{\CommentVarTok}[1]{\textcolor[rgb]{0.56,0.35,0.01}{\textbf{\textit{#1}}}}
\newcommand{\ConstantTok}[1]{\textcolor[rgb]{0.56,0.35,0.01}{#1}}
\newcommand{\ControlFlowTok}[1]{\textcolor[rgb]{0.13,0.29,0.53}{\textbf{#1}}}
\newcommand{\DataTypeTok}[1]{\textcolor[rgb]{0.13,0.29,0.53}{#1}}
\newcommand{\DecValTok}[1]{\textcolor[rgb]{0.00,0.00,0.81}{#1}}
\newcommand{\DocumentationTok}[1]{\textcolor[rgb]{0.56,0.35,0.01}{\textbf{\textit{#1}}}}
\newcommand{\ErrorTok}[1]{\textcolor[rgb]{0.64,0.00,0.00}{\textbf{#1}}}
\newcommand{\ExtensionTok}[1]{#1}
\newcommand{\FloatTok}[1]{\textcolor[rgb]{0.00,0.00,0.81}{#1}}
\newcommand{\FunctionTok}[1]{\textcolor[rgb]{0.13,0.29,0.53}{\textbf{#1}}}
\newcommand{\ImportTok}[1]{#1}
\newcommand{\InformationTok}[1]{\textcolor[rgb]{0.56,0.35,0.01}{\textbf{\textit{#1}}}}
\newcommand{\KeywordTok}[1]{\textcolor[rgb]{0.13,0.29,0.53}{\textbf{#1}}}
\newcommand{\NormalTok}[1]{#1}
\newcommand{\OperatorTok}[1]{\textcolor[rgb]{0.81,0.36,0.00}{\textbf{#1}}}
\newcommand{\OtherTok}[1]{\textcolor[rgb]{0.56,0.35,0.01}{#1}}
\newcommand{\PreprocessorTok}[1]{\textcolor[rgb]{0.56,0.35,0.01}{\textit{#1}}}
\newcommand{\RegionMarkerTok}[1]{#1}
\newcommand{\SpecialCharTok}[1]{\textcolor[rgb]{0.81,0.36,0.00}{\textbf{#1}}}
\newcommand{\SpecialStringTok}[1]{\textcolor[rgb]{0.31,0.60,0.02}{#1}}
\newcommand{\StringTok}[1]{\textcolor[rgb]{0.31,0.60,0.02}{#1}}
\newcommand{\VariableTok}[1]{\textcolor[rgb]{0.00,0.00,0.00}{#1}}
\newcommand{\VerbatimStringTok}[1]{\textcolor[rgb]{0.31,0.60,0.02}{#1}}
\newcommand{\WarningTok}[1]{\textcolor[rgb]{0.56,0.35,0.01}{\textbf{\textit{#1}}}}
\usepackage{graphicx}
\makeatletter
\def\maxwidth{\ifdim\Gin@nat@width>\linewidth\linewidth\else\Gin@nat@width\fi}
\def\maxheight{\ifdim\Gin@nat@height>\textheight\textheight\else\Gin@nat@height\fi}
\makeatother
% Scale images if necessary, so that they will not overflow the page
% margins by default, and it is still possible to overwrite the defaults
% using explicit options in \includegraphics[width, height, ...]{}
\setkeys{Gin}{width=\maxwidth,height=\maxheight,keepaspectratio}
% Set default figure placement to htbp
\makeatletter
\def\fps@figure{htbp}
\makeatother
\setlength{\emergencystretch}{3em} % prevent overfull lines
\providecommand{\tightlist}{%
  \setlength{\itemsep}{0pt}\setlength{\parskip}{0pt}}
\setcounter{secnumdepth}{-\maxdimen} % remove section numbering
\ifLuaTeX
\usepackage[bidi=basic]{babel}
\else
\usepackage[bidi=default]{babel}
\fi
\babelprovide[main,import]{spanish}
\ifPDFTeX
\else
\babelfont{rm}[]{Times New Roman}
\fi
% get rid of language-specific shorthands (see #6817):
\let\LanguageShortHands\languageshorthands
\def\languageshorthands#1{}
\ifLuaTeX
  \usepackage{selnolig}  % disable illegal ligatures
\fi
\usepackage{bookmark}
\IfFileExists{xurl.sty}{\usepackage{xurl}}{} % add URL line breaks if available
\urlstyle{same}
\hypersetup{
  pdflang={es-ES},
  hidelinks,
  pdfcreator={LaTeX via pandoc}}

\title{U8-ADMINISTRACIÓ D'UBUNTU (at i cron)}
\usepackage{etoolbox}
\makeatletter
\providecommand{\subtitle}[1]{% add subtitle to \maketitle
  \apptocmd{\@title}{\par {\large #1 \par}}{}{}
}
\makeatother
\subtitle{~Programar tasques. \textbf{at i cron}}
\author{}
\date{\vspace{-2.5em}}

\begin{document}
\maketitle

\setstretch{1.5}
\section{Gestió de Quotes de Disc i Còpies de Seguretat a
Ubuntu}\label{gestiuxf3-de-quotes-de-disc-i-cuxf2pies-de-seguretat-a-ubuntu}

\subsection{1. Quotes de disc en Linux
(Ubuntu)}\label{quotes-de-disc-en-linux-ubuntu}

Les \textbf{quotes de disc} permeten limitar l'espai d'emmagatzematge
que un usuari o grup pot utilitzar en un sistema de fitxers. Això és
útil per evitar que un sol usuari acapari tot l'espai disponible.

\subsubsection{1.1. Instal·lació de les quotes de
disc}\label{installaciuxf3-de-les-quotes-de-disc}

Ubuntu ve amb suport per a quotes, però cal activar-lo i configurar-lo.

\begin{enumerate}
\def\labelenumi{\arabic{enumi}.}
\item
  \textbf{Instal·lar el paquet de quotes (si no està instal·lat)}

\begin{Shaded}
\begin{Highlighting}[]
\FunctionTok{sudo}\NormalTok{ apt update }\KeywordTok{\&\&} \FunctionTok{sudo}\NormalTok{ apt install quota quotatool}
\end{Highlighting}
\end{Shaded}
\item
  \textbf{Activar les quotes en un sistema de fitxers}

  \begin{itemize}
  \item
    Edita \texttt{/etc/fstab} i afegeix \texttt{usrquota} i/o
    \texttt{grpquota} a la partició on vols aplicar quotes.\\
  \item
    Exemple per a \texttt{/home}:

\begin{verbatim}
/dev/sdX /home ext4 defaults,usrquota,grpquota 0 2
\end{verbatim}
  \end{itemize}
\item
  \textbf{Reiniciar per aplicar els canvis}

\begin{Shaded}
\begin{Highlighting}[]
\FunctionTok{sudo}\NormalTok{ reboot}
\end{Highlighting}
\end{Shaded}
\item
  \textbf{Crear els fitxers de quota}

\begin{Shaded}
\begin{Highlighting}[]
\FunctionTok{sudo}\NormalTok{ quotacheck }\AttributeTok{{-}cum}\NormalTok{ /home}
\FunctionTok{sudo}\NormalTok{ quotaon /home}
\end{Highlighting}
\end{Shaded}
\item
  \textbf{Establir quotes per a un usuari (exemple: usuari ``jordi'')}

\begin{Shaded}
\begin{Highlighting}[]
\FunctionTok{sudo}\NormalTok{ edquota }\AttributeTok{{-}u}\NormalTok{ jordi}
\end{Highlighting}
\end{Shaded}

  Això obrirà un editor on pots definir límits ``soft'' (avís) i
  ``hard'' (límits estrictes). Exemple:

\begin{verbatim}
Disk quotas for user jordi:
  Filesystem blocks soft hard inodes soft hard
  /dev/sdX  50000 40000 60000 0 0
\end{verbatim}
\item
  \textbf{Comprovar les quotes aplicades}

\begin{Shaded}
\begin{Highlighting}[]
\ExtensionTok{quota} \AttributeTok{{-}u}\NormalTok{ jordi}
\end{Highlighting}
\end{Shaded}
\item
  \textbf{Configurar un avís si un usuari excedeix el límit}

\begin{Shaded}
\begin{Highlighting}[]
\FunctionTok{sudo}\NormalTok{ warnquota}
\end{Highlighting}
\end{Shaded}
\end{enumerate}

\begin{center}\rule{0.5\linewidth}{0.5pt}\end{center}

\subsection{2. Còpies de Seguretat a
Ubuntu}\label{cuxf2pies-de-seguretat-a-ubuntu}

Fer còpies de seguretat és essencial per evitar pèrdues de dades. Ubuntu
ofereix diverses eines, com \textbf{rsync, tar, Timeshift, Déjà Dup,
Bacula}, entre d'altres.

\subsubsection{\texorpdfstring{2.1. Còpies de seguretat amb
\texttt{rsync}}{2.1. Còpies de seguretat amb rsync}}\label{cuxf2pies-de-seguretat-amb-rsync}

\texttt{rsync} permet sincronitzar directoris i fitxers de manera
eficient.

\begin{itemize}
\item
  \textbf{Còpia local}

\begin{Shaded}
\begin{Highlighting}[]
\FunctionTok{rsync} \AttributeTok{{-}av} \AttributeTok{{-}{-}delete}\NormalTok{ /home/jordi /mnt/backup/}
\end{Highlighting}
\end{Shaded}
\item
  \textbf{Còpia remota via SSH}

\begin{Shaded}
\begin{Highlighting}[]
\FunctionTok{rsync} \AttributeTok{{-}avz} \AttributeTok{{-}e}\NormalTok{ ssh /home/jordi usuario@server:/mnt/backup/}
\end{Highlighting}
\end{Shaded}
\item
  \textbf{Automatització amb \texttt{cron}}

  \begin{enumerate}
  \def\labelenumi{\arabic{enumi}.}
  \item
    Edita el crontab:

\begin{Shaded}
\begin{Highlighting}[]
\FunctionTok{crontab} \AttributeTok{{-}e}
\end{Highlighting}
\end{Shaded}
  \item
    Afegeix aquesta línia per fer una còpia cada nit a les 2:00 AM:

\begin{verbatim}
0 2 * * * rsync -av --delete /home/jordi /mnt/backup/
\end{verbatim}
  \end{enumerate}
\end{itemize}

\begin{center}\rule{0.5\linewidth}{0.5pt}\end{center}

\subsubsection{\texorpdfstring{2.2. Còpies amb
\texttt{tar}}{2.2. Còpies amb tar}}\label{cuxf2pies-amb-tar}

Si vols fer còpies comprimides, \texttt{tar} és una bona opció:

\begin{Shaded}
\begin{Highlighting}[]
\FunctionTok{tar} \AttributeTok{{-}czvf}\NormalTok{ backup\_home.tar.gz /home/jordi}
\end{Highlighting}
\end{Shaded}

Per restaurar:

\begin{Shaded}
\begin{Highlighting}[]
\FunctionTok{tar} \AttributeTok{{-}xzvf}\NormalTok{ backup\_home.tar.gz }\AttributeTok{{-}C}\NormalTok{ /}
\end{Highlighting}
\end{Shaded}

\begin{center}\rule{0.5\linewidth}{0.5pt}\end{center}

\subsubsection{2.3. Còpies automàtiques amb Déjà
Dup}\label{cuxf2pies-automuxe0tiques-amb-duxe9juxe0-dup}

Si prefereixes una interfície gràfica, \texttt{Déjà\ Dup} (Backup) és
una eina fàcil d'usar.

\begin{enumerate}
\def\labelenumi{\arabic{enumi}.}
\item
  Instal·lar Déjà Dup:

\begin{Shaded}
\begin{Highlighting}[]
\FunctionTok{sudo}\NormalTok{ apt install deja{-}dup}
\end{Highlighting}
\end{Shaded}
\item
  Obrir l'aplicació \textbf{Còpia de Seguretat} i configurar la
  destinació (disc extern, NAS, etc.).
\end{enumerate}

\end{document}
