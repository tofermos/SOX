% Options for packages loaded elsewhere
\PassOptionsToPackage{unicode}{hyperref}
\PassOptionsToPackage{hyphens}{url}
%
\documentclass[
  12 pt,
  a4paper,
]{article}
\usepackage{amsmath,amssymb}
\usepackage{setspace}
\usepackage{iftex}
\ifPDFTeX
  \usepackage[T1]{fontenc}
  \usepackage[utf8]{inputenc}
  \usepackage{textcomp} % provide euro and other symbols
\else % if luatex or xetex
  \usepackage{unicode-math} % this also loads fontspec
  \defaultfontfeatures{Scale=MatchLowercase}
  \defaultfontfeatures[\rmfamily]{Ligatures=TeX,Scale=1}
\fi
\usepackage{lmodern}
\ifPDFTeX\else
  % xetex/luatex font selection
  \setmainfont[]{Times New Roman}
\fi
% Use upquote if available, for straight quotes in verbatim environments
\IfFileExists{upquote.sty}{\usepackage{upquote}}{}
\IfFileExists{microtype.sty}{% use microtype if available
  \usepackage[]{microtype}
  \UseMicrotypeSet[protrusion]{basicmath} % disable protrusion for tt fonts
}{}
\makeatletter
\@ifundefined{KOMAClassName}{% if non-KOMA class
  \IfFileExists{parskip.sty}{%
    \usepackage{parskip}
  }{% else
    \setlength{\parindent}{0pt}
    \setlength{\parskip}{6pt plus 2pt minus 1pt}}
}{% if KOMA class
  \KOMAoptions{parskip=half}}
\makeatother
\usepackage{xcolor}
\usepackage[margin=1in]{geometry}
\usepackage{color}
\usepackage{fancyvrb}
\newcommand{\VerbBar}{|}
\newcommand{\VERB}{\Verb[commandchars=\\\{\}]}
\DefineVerbatimEnvironment{Highlighting}{Verbatim}{commandchars=\\\{\}}
% Add ',fontsize=\small' for more characters per line
\usepackage{framed}
\definecolor{shadecolor}{RGB}{248,248,248}
\newenvironment{Shaded}{\begin{snugshade}}{\end{snugshade}}
\newcommand{\AlertTok}[1]{\textcolor[rgb]{0.94,0.16,0.16}{#1}}
\newcommand{\AnnotationTok}[1]{\textcolor[rgb]{0.56,0.35,0.01}{\textbf{\textit{#1}}}}
\newcommand{\AttributeTok}[1]{\textcolor[rgb]{0.13,0.29,0.53}{#1}}
\newcommand{\BaseNTok}[1]{\textcolor[rgb]{0.00,0.00,0.81}{#1}}
\newcommand{\BuiltInTok}[1]{#1}
\newcommand{\CharTok}[1]{\textcolor[rgb]{0.31,0.60,0.02}{#1}}
\newcommand{\CommentTok}[1]{\textcolor[rgb]{0.56,0.35,0.01}{\textit{#1}}}
\newcommand{\CommentVarTok}[1]{\textcolor[rgb]{0.56,0.35,0.01}{\textbf{\textit{#1}}}}
\newcommand{\ConstantTok}[1]{\textcolor[rgb]{0.56,0.35,0.01}{#1}}
\newcommand{\ControlFlowTok}[1]{\textcolor[rgb]{0.13,0.29,0.53}{\textbf{#1}}}
\newcommand{\DataTypeTok}[1]{\textcolor[rgb]{0.13,0.29,0.53}{#1}}
\newcommand{\DecValTok}[1]{\textcolor[rgb]{0.00,0.00,0.81}{#1}}
\newcommand{\DocumentationTok}[1]{\textcolor[rgb]{0.56,0.35,0.01}{\textbf{\textit{#1}}}}
\newcommand{\ErrorTok}[1]{\textcolor[rgb]{0.64,0.00,0.00}{\textbf{#1}}}
\newcommand{\ExtensionTok}[1]{#1}
\newcommand{\FloatTok}[1]{\textcolor[rgb]{0.00,0.00,0.81}{#1}}
\newcommand{\FunctionTok}[1]{\textcolor[rgb]{0.13,0.29,0.53}{\textbf{#1}}}
\newcommand{\ImportTok}[1]{#1}
\newcommand{\InformationTok}[1]{\textcolor[rgb]{0.56,0.35,0.01}{\textbf{\textit{#1}}}}
\newcommand{\KeywordTok}[1]{\textcolor[rgb]{0.13,0.29,0.53}{\textbf{#1}}}
\newcommand{\NormalTok}[1]{#1}
\newcommand{\OperatorTok}[1]{\textcolor[rgb]{0.81,0.36,0.00}{\textbf{#1}}}
\newcommand{\OtherTok}[1]{\textcolor[rgb]{0.56,0.35,0.01}{#1}}
\newcommand{\PreprocessorTok}[1]{\textcolor[rgb]{0.56,0.35,0.01}{\textit{#1}}}
\newcommand{\RegionMarkerTok}[1]{#1}
\newcommand{\SpecialCharTok}[1]{\textcolor[rgb]{0.81,0.36,0.00}{\textbf{#1}}}
\newcommand{\SpecialStringTok}[1]{\textcolor[rgb]{0.31,0.60,0.02}{#1}}
\newcommand{\StringTok}[1]{\textcolor[rgb]{0.31,0.60,0.02}{#1}}
\newcommand{\VariableTok}[1]{\textcolor[rgb]{0.00,0.00,0.00}{#1}}
\newcommand{\VerbatimStringTok}[1]{\textcolor[rgb]{0.31,0.60,0.02}{#1}}
\newcommand{\WarningTok}[1]{\textcolor[rgb]{0.56,0.35,0.01}{\textbf{\textit{#1}}}}
\usepackage{graphicx}
\makeatletter
\def\maxwidth{\ifdim\Gin@nat@width>\linewidth\linewidth\else\Gin@nat@width\fi}
\def\maxheight{\ifdim\Gin@nat@height>\textheight\textheight\else\Gin@nat@height\fi}
\makeatother
% Scale images if necessary, so that they will not overflow the page
% margins by default, and it is still possible to overwrite the defaults
% using explicit options in \includegraphics[width, height, ...]{}
\setkeys{Gin}{width=\maxwidth,height=\maxheight,keepaspectratio}
% Set default figure placement to htbp
\makeatletter
\def\fps@figure{htbp}
\makeatother
\setlength{\emergencystretch}{3em} % prevent overfull lines
\providecommand{\tightlist}{%
  \setlength{\itemsep}{0pt}\setlength{\parskip}{0pt}}
\setcounter{secnumdepth}{-\maxdimen} % remove section numbering
\ifLuaTeX
\usepackage[bidi=basic]{babel}
\else
\usepackage[bidi=default]{babel}
\fi
\babelprovide[main,import]{spanish}
\ifPDFTeX
\else
\babelfont{rm}[]{Times New Roman}
\fi
% get rid of language-specific shorthands (see #6817):
\let\LanguageShortHands\languageshorthands
\def\languageshorthands#1{}
\ifLuaTeX
  \usepackage{selnolig}  % disable illegal ligatures
\fi
\usepackage{bookmark}
\IfFileExists{xurl.sty}{\usepackage{xurl}}{} % add URL line breaks if available
\urlstyle{same}
\hypersetup{
  pdflang={es-ES},
  hidelinks,
  pdfcreator={LaTeX via pandoc}}

\title{U9-ADMINISTRACIÓ D'UBUNTU. Monitorització}
\usepackage{etoolbox}
\makeatletter
\providecommand{\subtitle}[1]{% add subtitle to \maketitle
  \apptocmd{\@title}{\par {\large #1 \par}}{}{}
}
\makeatother
\subtitle{~Processos}
\author{}
\date{\vspace{-2.5em}}

\begin{document}
\maketitle

\setstretch{1.5}
Quan envies un procés a \textbf{segon pla} (\texttt{background}),
\textbf{continua executant-se}, però ho fa \textbf{sense bloquejar la
terminal}, permetent-te executar altres ordres al mateix temps.

\subsection{Diferències clau:}\label{diferuxe8ncies-clau}

\begin{itemize}
\tightlist
\item
  \textbf{En segon pla (\texttt{\&}):} El procés està \textbf{en
  execució activa}, però \textbf{no ocupa el terminal principal}.\\
\item
  \textbf{Pausat (\texttt{Ctrl\ +\ Z}):} El procés està \textbf{aturat}
  fins que li dius \texttt{bg} o \texttt{fg}.
\end{itemize}

\subsection{Exemple pràctic per
comprovar-ho:}\label{exemple-pruxe0ctic-per-comprovar-ho}

\subsubsection{Envia'l a segon pla:}\label{envial-a-segon-pla}

\begin{Shaded}
\begin{Highlighting}[]
\FunctionTok{sleep}\NormalTok{ 30 }\KeywordTok{\&}
\end{Highlighting}
\end{Shaded}

Resultat:

\begin{verbatim}
[1] 12345
\end{verbatim}

\begin{itemize}
\tightlist
\item
  \texttt{{[}1{]}}: Número de treball (job number).\\
\item
  \texttt{12345}: PID (Identificador del procés).
\end{itemize}

\subsubsection{\texorpdfstring{Comprova si està executant-se amb
\texttt{jobs}:}{Comprova si està executant-se amb jobs:}}\label{comprova-si-estuxe0-executant-se-amb-jobs}

\begin{Shaded}
\begin{Highlighting}[]
\BuiltInTok{jobs}
\end{Highlighting}
\end{Shaded}

Resultat:

\begin{verbatim}
[1]+  Running              sleep 30 &
\end{verbatim}

\begin{itemize}
\tightlist
\item
  \textbf{``Running''}: Sí, està executant-se en segon pla.
\end{itemize}

\begin{center}\rule{0.5\linewidth}{0.5pt}\end{center}

\subsection{\texorpdfstring{Comprova amb \texttt{ps} si està
executant-se}{Comprova amb ps si està executant-se}}\label{comprova-amb-ps-si-estuxe0-executant-se}

\begin{Shaded}
\begin{Highlighting}[]
\FunctionTok{ps} \AttributeTok{{-}ef} \KeywordTok{|} \FunctionTok{grep}\NormalTok{ sleep}
\end{Highlighting}
\end{Shaded}

Veuràs una línia amb el \textbf{PID} del procés.

\begin{center}\rule{0.5\linewidth}{0.5pt}\end{center}

\subsection{Si el processos s'atura
(Stopped)}\label{si-el-processos-satura-stopped}

\begin{itemize}
\tightlist
\item
  Si veus \texttt{Stopped} a \texttt{jobs}:
\end{itemize}

\begin{Shaded}
\begin{Highlighting}[]
\ExtensionTok{[1]+}\NormalTok{  Stopped              sleep 30}
\end{Highlighting}
\end{Shaded}

Això passa si l'has aturat amb \texttt{Ctrl\ +\ Z}.

\begin{itemize}
\tightlist
\item
  Reprèn-lo en segon pla amb:
\end{itemize}

\begin{Shaded}
\begin{Highlighting}[]
\BuiltInTok{bg}\NormalTok{ \%1}
\end{Highlighting}
\end{Shaded}

\begin{center}\rule{0.5\linewidth}{0.5pt}\end{center}

\subsection{\texorpdfstring{Comprova si s'executa després de tancar el
terminal
(\texttt{nohup}):}{Comprova si s'executa després de tancar el terminal (nohup):}}\label{comprova-si-sexecuta-despruxe9s-de-tancar-el-terminal-nohup}

\begin{Shaded}
\begin{Highlighting}[]
\FunctionTok{nohup}\NormalTok{ ./el\_meu\_script.sh }\KeywordTok{\&}
\BuiltInTok{disown}\NormalTok{ \%1}
\end{Highlighting}
\end{Shaded}

\emph{disown} desvincula (sense deixar d'excutar-se) el procés del
terminal.

\begin{center}\rule{0.5\linewidth}{0.5pt}\end{center}

\subsection{Resum curt}\label{resum-curt}

\begin{itemize}
\tightlist
\item
  \texttt{\&} → Executa i \textbf{continua treballant} en segon pla.\\
\item
  \texttt{jobs} → Verifica si està en execució (\texttt{Running} o
  \texttt{Stopped}).\\
\item
  \texttt{ps} → Mostra processos actius.\\
\item
  \texttt{bg} → Continua en segon pla si estava pausat.\\
\item
  \texttt{nohup} → Continua després de tancar la terminal.
\end{itemize}

\end{document}
