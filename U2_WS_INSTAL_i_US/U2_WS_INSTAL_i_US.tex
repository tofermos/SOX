% Options for packages loaded elsewhere
\PassOptionsToPackage{unicode}{hyperref}
\PassOptionsToPackage{hyphens}{url}
%
\documentclass[
  a4paper,
]{article}
\usepackage{amsmath,amssymb}
\usepackage{setspace}
\usepackage{iftex}
\ifPDFTeX
  \usepackage[T1]{fontenc}
  \usepackage[utf8]{inputenc}
  \usepackage{textcomp} % provide euro and other symbols
\else % if luatex or xetex
  \usepackage{unicode-math} % this also loads fontspec
  \defaultfontfeatures{Scale=MatchLowercase}
  \defaultfontfeatures[\rmfamily]{Ligatures=TeX,Scale=1}
\fi
\usepackage{lmodern}
\ifPDFTeX\else
  % xetex/luatex font selection
\fi
% Use upquote if available, for straight quotes in verbatim environments
\IfFileExists{upquote.sty}{\usepackage{upquote}}{}
\IfFileExists{microtype.sty}{% use microtype if available
  \usepackage[]{microtype}
  \UseMicrotypeSet[protrusion]{basicmath} % disable protrusion for tt fonts
}{}
\makeatletter
\@ifundefined{KOMAClassName}{% if non-KOMA class
  \IfFileExists{parskip.sty}{%
    \usepackage{parskip}
  }{% else
    \setlength{\parindent}{0pt}
    \setlength{\parskip}{6pt plus 2pt minus 1pt}}
}{% if KOMA class
  \KOMAoptions{parskip=half}}
\makeatother
\usepackage{xcolor}
\usepackage[margin=1in]{geometry}
\usepackage{graphicx}
\makeatletter
\def\maxwidth{\ifdim\Gin@nat@width>\linewidth\linewidth\else\Gin@nat@width\fi}
\def\maxheight{\ifdim\Gin@nat@height>\textheight\textheight\else\Gin@nat@height\fi}
\makeatother
% Scale images if necessary, so that they will not overflow the page
% margins by default, and it is still possible to overwrite the defaults
% using explicit options in \includegraphics[width, height, ...]{}
\setkeys{Gin}{width=\maxwidth,height=\maxheight,keepaspectratio}
% Set default figure placement to htbp
\makeatletter
\def\fps@figure{htbp}
\makeatother
\setlength{\emergencystretch}{3em} % prevent overfull lines
\providecommand{\tightlist}{%
  \setlength{\itemsep}{0pt}\setlength{\parskip}{0pt}}
\setcounter{secnumdepth}{-\maxdimen} % remove section numbering
\ifLuaTeX
\usepackage[bidi=basic]{babel}
\else
\usepackage[bidi=default]{babel}
\fi
\babelprovide[main,import]{catalan}
% get rid of language-specific shorthands (see #6817):
\let\LanguageShortHands\languageshorthands
\def\languageshorthands#1{}
\ifLuaTeX
  \usepackage{selnolig}  % disable illegal ligatures
\fi
\usepackage{bookmark}
\IfFileExists{xurl.sty}{\usepackage{xurl}}{} % add URL line breaks if available
\urlstyle{same}
\hypersetup{
  pdftitle={U2. Windows Server. Instal·lació i ús},
  pdfauthor={@tofermos 2024},
  pdflang={ca-ES},
  hidelinks,
  pdfcreator={LaTeX via pandoc}}

\title{U2. Windows Server. Instal·lació i ús}
\author{@tofermos 2024}
\date{}

\begin{document}
\maketitle

{
\setcounter{tocdepth}{2}
\tableofcontents
}
\setstretch{1.5}
\newpage
\renewcommand\tablename{Tabla}

\section{1. Introducció a Windows
Server}\label{introducciuxf3-a-windows-server}

Windows Server és un sistema operatiu desenvolupat per Microsoft,
dissenyat per a administrar xarxes, emmagatzematge i aplicacions a
nivell empresarial. A diferència de les versions de Windows orientades a
usuaris individuals, Windows Server està optimitzat per a la gestió de
serveis en xarxa, com ara allotjament de llocs web, gestió de bases de
dades i centralització de recursos compartits.

Algunes de les principals característiques de Windows Server inclouen:

\begin{itemize}
\item
  \textbf{Active Directory}: Un \emph{servei de directori} que permet la
  gestió d'usuaris, equips i polítiques de seguretat en la xarxa de
  forma centralitzada.
\item
  \textbf{Hyper-V}: Tecnologia de virtualització integrada que permet
  crear i administrar màquines virtuals.
\item
  \textbf{IIS (Internet Information Services)}: Plataforma per a
  allotjar aplicacions web. Pemet que el servidor siga un servidor web.
\item
  \textbf{Administració de servidors}: A través d'eines com
  l'Administrador de Servidors, es poden gestionar múltiples rols i
  funcions del servidor.
\end{itemize}

\section{2. Instal·lació en un equip
informàtic}\label{installaciuxf3-en-un-equip-informuxe0tic}

Abans d'utilitzar Windows Server, és essencial entendre el procés
d'instal·lació, que consta de diversos passos tècnics importants.
Aquests asseguren que el sistema operatiu funcione correctament i que
estiga ben integrat amb el maquinari i els recursos de la xarxa.

No obstant, no són molt diferents al ja vistos en un Windows 1x.

\subsection{2.1 Particions i sistemes
d'arxius}\label{particions-i-sistemes-darxius}

Les particions són divisions lògiques del disc dur on s'emmagatzemarà el
sistema operatiu i les dades. Durant la instal·lació de Windows Server,
cal triar com particionar el disc.

\begin{itemize}
\item
  \textbf{Partició primària}: És on s'instal·la el sistema operatiu. Ha
  de ser activa per a poder arrancar des d'ella.
\item
  \textbf{Sistema d'arxius}: Windows Server utilitza principalment el
  sistema \textbf{NTFS (New Technology File System)}, que ofereix
  característiques avançades com permisos d'arxiu, xifratge i major
  eficiència en la gestió de l'espai en disc. Els permisos són un
  aspecte més important que tractarem al curs.
\end{itemize}

A banda de superar les limitacions com ja sabeu del curs passat de FAT32
i facilita d'accès per a Linux.

\subsubsection{Taula de particions}\label{taula-de-particions}

El nombre de particions possibles i el tamany màxim varien segons si el
disc utilitza el \textbf{GPT} (GUID Partition Table) o el \textbf{MBR}
(Master Boot Record):

\textbf{MBR (Master Boot Record)} - \textbf{Nombre màxim de particions:}
4 particions primàries. Si es necessita més de 4 particions, es poden
tenir 3 particions primàries i 1 partició estesa, dins la qual es poden
crear múltiples particions lògiques. - \textbf{Tipus de particions:} -
\textbf{Primàries:} Particions directament accessibles pel sistema
operatiu (màxim 4). - \textbf{Estesa:} Una partició que pot contenir
particions lògiques. - \textbf{Lògiques:} Particions dins de la partició
estesa. - \textbf{Tamany màxim del disc}: 2 TB (terabytes), el MBR
utilitza 32 bits per adreçar sectors de 512 Bytes.

\subsubsection{\texorpdfstring{\textbf{GPT (GUID Partition
Table)}}{GPT (GUID Partition Table)}}\label{gpt-guid-partition-table}

\begin{itemize}
\tightlist
\item
  \textbf{Nombre màxim de particions:} Teòricament il·limitat, però
  pràcticament uns 128 en la majoria de sistemes operatius.
\item
  \textbf{Tipus de particions:}

  \begin{itemize}
  \tightlist
  \item
    Totes les particions són \textbf{primàries}, no hi ha limitació
    d'estesa o lògiques com en MBR.
  \end{itemize}
\item
  \textbf{Nombre màxim de particions:} Teòricament il·limitat, però
  pràcticament uns 128 en la majoria de sistemes operatius.
\item
  \textbf{Tipus de particions:}

  \begin{itemize}
  \tightlist
  \item
    Totes les particions són \textbf{primàries}
  \end{itemize}
\item
  \textbf{Tamany màxim del disc} 9.4 ZB (zettabytes) teòrics. GPT
  utilitza un esquema d'adreces de 64 bits
\end{itemize}

\subsubsection{Recomanacions generals}\label{recomanacions-generals}

\begin{enumerate}
\def\labelenumi{\arabic{enumi}.}
\tightlist
\item
  Usar GPT preferentment.
\item
  Separar en discos durs distints la instal·lació del SO de la resta de
  dades.
\item
  Si només tenim un disc (o conjunt que implementes una unitat tipus
  RAID), crearem particions distintes.
\item
  Si potser les crearem en el procés d'intal·lació o, almenys, abans de
  començar a operar. En cas contrari, hem de fer còpia de seguretat de
  les dades abans de reparticionar discos.
\item
  Si tenim discos de distint rendiment (SSD i HDD), és preferible que el
  SO i les aplicacions més utilitzades o de més demanda de capacitat de
  processament estiguen instal·lades al disc més ràpid.
\item
  Tant el backup com els sitems de redundància són fonamentals. (Els
  veiem més avant)
\end{enumerate}

\subsection{2.2 Gestors d'arrancada}\label{gestors-darrancada}

El gestor d'arrancada és un programari que s'encarrega d'iniciar el
sistema operatiu durant el procés d'arrancada de l'equip.

\begin{itemize}
\tightlist
\item
  \textbf{Windows Boot Manager} és el gestor d'arrancada predeterminat
  de Windows Server. Aquest programari s'encarrega de gestionar el
  procés d'arrancada del sistema i, en cas d'haver-hi diversos sistemes
  operatius instal·lats en l'equip, permet seleccionar quin iniciar.
\end{itemize}

\subsection{2.3 Procés d'instal·lació}\label{procuxe9s-dinstallaciuxf3}

El procés d'instal·lació de Windows Server és similar al de les versions
d'escriptori de Windows, però amb alguns passos addicionals per a la
configuració de rols i característiques del servidor. Els passos
generals inclouen:

\begin{enumerate}
\def\labelenumi{\arabic{enumi}.}
\tightlist
\item
  \textbf{Preparar l'equip}: Comprovar els requisits de maquinari, com
  CPU, RAM i espai en disc.
\item
  \textbf{Arrancada des del mitjà d'instal·lació}: Açò pot ser des d'un
  DVD o una unitat USB bootable. Al VirtualBox usarem la unitat òptica.
\item
  \textbf{Selecció de la partició}: Triar la unitat o partició on
  s'instal·larà Windows Server.
\item
  \textbf{Configuració inicial}: Assignar un nom al servidor, configurar
  la xarxa i el compte d'administrador. Serà administrador de tota la
  xarxa com vorem.
\item
  \textbf{Configuració de rols}: Durant o després de la instal·lació, es
  poden afegir rols al servidor. El més important és el de controlador
  de domini però hi ha altres.
\end{enumerate}

Aquests darres aspectes el veiem de forma pràctica en al document
següent.

\section{3. Utilització de Windows
Server}\label{utilitzaciuxf3-de-windows-server}

Una vegada instal·lat, és essencial familiaritzar-se amb la interfície i
les funcionalitats bàsiques de Windows Server per a poder administrar-lo
correctament.

\subsection{3.1 Conceptes generals. Rols i
característiques.}\label{conceptes-generals.-rols-i-caracteruxedstiques.}

\subsubsection{Rols}\label{rols}

Els rols són funcionalitats principals que pot exercir el servidor. En
\textbf{Windows Server} hi ha diversos \textbf{rols de servidor} que
poden ser instal·lats per oferir serveis específics a una xarxa o
organització.

Rols principals de \textbf{Windows Server}:

\begin{enumerate}
\def\labelenumi{\arabic{enumi}.}
\item
  \textbf{Active Directory Domain Services (AD DS)}: Permet crear i
  gestionar dominis, i és el component clau de l'Active Directory per a
  la gestió d'usuaris i dispositius en una xarxa.
\item
  \textbf{DHCP Server}: Assigna automàticament adreces IP als
  dispositius de la xarxa.
\item
  \textbf{DNS Server}: Traduïx noms de domini a adreces IP, facilitant
  l'accés als serveis dins d'una xarxa o a internet.
\item
  \textbf{File and Storage Services}: Gestiona el sistema
  d'emmagatzematge de fitxers i carpetes compartides, i permet utilitzar
  funcions com el servidor de fitxers, les quotes d'emmagatzematge i la
  deduplicació de dades.
\item
  \textbf{Remote Desktop Services (RDS)}: Proporciona eines per permetre
  que els usuaris es connecten de forma remota a escriptoris virtuals o
  aplicacions publicades.
\item
  \textbf{Print and Document Services}: Permet gestionar impressores i
  compartir-les en la xarxa.
\item
  \textbf{Hyper-V}: Proporciona funcions de virtualització per crear i
  gestionar màquines virtuals.
\item
  \textbf{Web Server (IIS)}: Hosteja aplicacions web i llocs web
  utilitzant \textbf{Internet Information Services (IIS)}.
\item
  \textbf{Network Policy and Access Services (NPAS)}: Permet la gestió
  de polítiques de xarxa i l'accés remot mitjançant serveis com el
  Servidor RADIUS.
\item
  \textbf{Windows Deployment Services (WDS)}: Serveix per desplegar de
  forma remota sistemes operatius als ordinadors clients a través de la
  xarxa.
\item
  \textbf{Active Directory Certificate Services (AD CS)}: Gestiona
  certificats digitals per a l'autenticació, la xifratge i la seguretat
  a la xarxa.
\item
  \textbf{Active Directory Federation Services (AD FS)}: Proporciona
  serveis d'inici de sessió únic (SSO) per a aplicacions.
\item
  \textbf{Active Directory Rights Management Services (AD RMS)}:
  Protegeix informació mitjançant polítiques de drets d'accés.
\item
  \textbf{Failover Clustering}: Proporciona alta disponibilitat a
  serveis i aplicacions mitjançant la creació de clústers de servidors.
\item
  \textbf{Windows Server Update Services (WSUS)}: Gestió de les
  actualitzacions de seguretat i programari per als clients de la xarxa.
\item
  \textbf{Host Guardian Service}: Ofereix protecció avançada per a
  entorns Hyper-V amb màquines virtuals protegides.
\item
  \textbf{Network Load Balancing (NLB)}: Permet distribuir el trànsit de
  xarxa entre múltiples servidors per assegurar alta disponibilitat i
  escalabilitat.
\end{enumerate}

\emph{En aquest curs ens centrerem en els 5 primers que són els
fonamentals en qualsevol LAN}

\subsubsection{Característiques}\label{caracteruxedstiques}

A més dels rols, \textbf{Windows Server} ofereix
\textbf{característiques} que són complements als rols i que
proporcionen funcionalitats addicionals:

\begin{enumerate}
\def\labelenumi{\arabic{enumi}.}
\item
  \textbf{Windows PowerShell}. Llenguatge d'scripts basat em ordre
  (cmdLets) molt potent i avançat.
\item
  \textbf{Servici de backup de Seguretat de Windows Server}. El vorem.
\item
  \textbf{.NET Framework} (no el vorem)
\item
  \textbf{Windows Defender}
\item
  \textbf{Remote assistence}
\end{enumerate}

Com veiem hi ha una clara relació entre el roll i la classificació dels
servidors segons la funció que vam estudiar a la unitat anterior.

Per altra banda, la característica ve a complementar el roll o facilitar
la tasca de l'administrador.

\subsection{3.2 Llicències}\label{llicuxe8ncies}

Windows Server requereix llicències tant per al sistema operatiu com per
als usuaris o dispositius que accedeixen al servidor (CALs - Client
Access Licenses).

\subsection{3.2 Interfície gràfica}\label{interfuxedcie-gruxe0fica}

La interfície gràfica de Windows Server és similar a la de les versions
d'escriptori de Windows, però està optimitzada per a l'administració de
servidors. Algunes característiques clau inclouen:

Al \textbf{curs de Windows 11} d'aquest mateix repositori podreu trobar
informació sobre l'entorn gràfic comú de tots els Windows per a la
gestió. Estudieu-ho.

\href{https://tofermos.github.io/Windows11/interfaces/interfaces.html}{Introducció
a l'entorn gràfic de Windows}

\href{https://tofermos.github.io/Windows11/gestiodelequip/gestiodelequip.html}{Gestió
des de l'entorn gràfic Windows}

\subsubsection{Sobre consoles i altres utilitats de
gestió.}\label{sobre-consoles-i-altres-utilitats-de-gestiuxf3.}

A bande de les vistes en l'apartat anterior i que són comunes, la
pràctica totalitat, a tots els Windows tenim que,específicament de
Windows Server les consoles i utilitats següents:

\textbf{servermanager.exe} - Administrador de Servidors. Aquesta és la
utilitat (no es consola estrictament parlant) central per a gestionar el
servidor. Permet configurar rols i característiques, gestionar discos,
supervisar el rendiment, entre altres funcions.

\textbf{dcpromo.msc} - Promoció de controlador de domini: Utilitzada per
configurar un controlador de domini (AD DS), una funció exclusiva de
Windows Server.

\textbf{dnsmgmt.msc} - Gestió de DNS: Disponible en Windows Server per
gestionar zones i registres DNS.

\textbf{dhcpmgmt.msc} - Gestió de DHCP: Permet administrar el rol de
servidor DHCP per assignar adreces IP automàticament a dispositius de la
xarxa.

\textbf{fsmgmt.msc} - Carpetes compartides: Una consola específica per
gestionar carpetes i recursos compartits al servidor, encara que també
es pot trobar en versions professionals de Windows 10/11.

\textbf{tsadmin.msc} o Remote Desktop Services Manager: Utilitzada per
gestionar sessions d'escriptori remot, més comuna en Windows Server per
administrar entorns d'escriptori remot (RDS).

\textbf{cluadmin.msc} - Gestió de Clúster de Failover: Disponible en
Windows Server per administrar clústers de tolerància a fallades i alta
disponibilitat, especialment útil per entorns crítics empresarials.

\subsection{\texorpdfstring{3.3 \textbf{Tancament de sessió i apagat del
servidor}}{3.3 Tancament de sessió i apagat del servidor}}\label{tancament-de-sessiuxf3-i-apagat-del-servidor}

El tancament de sessió i l'apagat d'un servidor requereix més
precaucions que en una estació de treball (Windows 1x).

Respecte a l'\textbf{apagat del servidor}, aquets, solen estar executant
serveis crítics i aplicacions de xarxa. Apagar incorrectament un
servidor pot causar pèrdua de dades o interrupció del servei. El procés
d'apagat ha de ser planificat i queda registrada la causa que indiquem.

Respecte al \textbf{tancament de sessió}, l'estat normal deu ser amb la
sessió tancada excepte quan un administrador ha de fer alguna tasca. La
sessió deu estar el temps estrictament necessari i el login no deu estar
visible per raons de seguretat.

\end{document}
