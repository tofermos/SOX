% Options for packages loaded elsewhere
\PassOptionsToPackage{unicode}{hyperref}
\PassOptionsToPackage{hyphens}{url}
%
\documentclass[
  a4paper,
]{article}
\usepackage{amsmath,amssymb}
\usepackage{setspace}
\usepackage{iftex}
\ifPDFTeX
  \usepackage[T1]{fontenc}
  \usepackage[utf8]{inputenc}
  \usepackage{textcomp} % provide euro and other symbols
\else % if luatex or xetex
  \usepackage{unicode-math} % this also loads fontspec
  \defaultfontfeatures{Scale=MatchLowercase}
  \defaultfontfeatures[\rmfamily]{Ligatures=TeX,Scale=1}
\fi
\usepackage{lmodern}
\ifPDFTeX\else
  % xetex/luatex font selection
\fi
% Use upquote if available, for straight quotes in verbatim environments
\IfFileExists{upquote.sty}{\usepackage{upquote}}{}
\IfFileExists{microtype.sty}{% use microtype if available
  \usepackage[]{microtype}
  \UseMicrotypeSet[protrusion]{basicmath} % disable protrusion for tt fonts
}{}
\makeatletter
\@ifundefined{KOMAClassName}{% if non-KOMA class
  \IfFileExists{parskip.sty}{%
    \usepackage{parskip}
  }{% else
    \setlength{\parindent}{0pt}
    \setlength{\parskip}{6pt plus 2pt minus 1pt}}
}{% if KOMA class
  \KOMAoptions{parskip=half}}
\makeatother
\usepackage{xcolor}
\usepackage[margin=1in]{geometry}
\usepackage{graphicx}
\makeatletter
\def\maxwidth{\ifdim\Gin@nat@width>\linewidth\linewidth\else\Gin@nat@width\fi}
\def\maxheight{\ifdim\Gin@nat@height>\textheight\textheight\else\Gin@nat@height\fi}
\makeatother
% Scale images if necessary, so that they will not overflow the page
% margins by default, and it is still possible to overwrite the defaults
% using explicit options in \includegraphics[width, height, ...]{}
\setkeys{Gin}{width=\maxwidth,height=\maxheight,keepaspectratio}
% Set default figure placement to htbp
\makeatletter
\def\fps@figure{htbp}
\makeatother
\setlength{\emergencystretch}{3em} % prevent overfull lines
\providecommand{\tightlist}{%
  \setlength{\itemsep}{0pt}\setlength{\parskip}{0pt}}
\setcounter{secnumdepth}{-\maxdimen} % remove section numbering
\ifLuaTeX
\usepackage[bidi=basic]{babel}
\else
\usepackage[bidi=default]{babel}
\fi
\babelprovide[main,import]{catalan}
% get rid of language-specific shorthands (see #6817):
\let\LanguageShortHands\languageshorthands
\def\languageshorthands#1{}
\ifLuaTeX
  \usepackage{selnolig}  % disable illegal ligatures
\fi
\usepackage{bookmark}
\IfFileExists{xurl.sty}{\usepackage{xurl}}{} % add URL line breaks if available
\urlstyle{same}
\hypersetup{
  pdftitle={U2.Instal·lació i ús de Windows Server},
  pdfauthor={@tofermos 2024},
  pdflang={ca-ES},
  hidelinks,
  pdfcreator={LaTeX via pandoc}}

\title{U2.Instal·lació i ús de Windows Server}
\usepackage{etoolbox}
\makeatletter
\providecommand{\subtitle}[1]{% add subtitle to \maketitle
  \apptocmd{\@title}{\par {\large #1 \par}}{}{}
}
\makeatother
\subtitle{Activitat Unitat 2}
\author{@tofermos 2024}
\date{}

\begin{document}
\maketitle

{
\setcounter{tocdepth}{2}
\tableofcontents
}
\setstretch{1.5}
\newpage
\renewcommand\tablename{Tabla}

\begin{quote}
NOTA PRÈVIA:

Abans de començar, estudieu la Unitat 2 i llegiu TOTA la pràctica. Hi
pot haver un apartat que no pugues contestar fins el final o una dada
que te la proporcionen després.
\end{quote}

\section{Descripció}\label{descripciuxf3}

No aneu a cerar encara un domini però tots sí tots els primers passos
per a fer-ho.

\subsection{Pas 1:}\label{pas-1}

L'objectiu de la activitat és connectar un Windows 1x amb un Windows
Server en un Workgroup de forma similar a la pràctica anterior però
caldrà, a més :

\begin{itemize}
\tightlist
\item
  Indiqueu la IP de DNS en la configuració del adaptadors.
\item
  Assegureu-vos la compartició i detecció de xarxes que suposa més faena
  (repassa el tema) però Sense deixar un forat en la seguretat.
\item
  Configureu altres qüestions del Servidor tal com heu vist a la Unitat
  2 com el Windows Update i indiqueu perquè.
\end{itemize}

\subsection{Pas 2:}\label{pas-2}

Una vegada el WG funcionant, creeu una carpeta compartida en les dos
màquines i assigneu una unitat de xarxa.

\begin{itemize}
\item
  En el client, l'assignareu des del GUI (a la carpeta del servidor).
\item
  En el servidor, l'assignareu des del CLI (a la carpeta del client)
\end{itemize}

Crea un fitxr txt (amb el Bloc de Notas, per exemple) en la carpeta
compartida del servidor i obri'l des del client. Força el seu tantcament
des del servidor.

\end{document}
