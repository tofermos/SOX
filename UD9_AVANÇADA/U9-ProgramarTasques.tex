% Options for packages loaded elsewhere
\PassOptionsToPackage{unicode}{hyperref}
\PassOptionsToPackage{hyphens}{url}
%
\documentclass[
  12 pt,
  a4paper,
]{article}
\usepackage{amsmath,amssymb}
\usepackage{setspace}
\usepackage{iftex}
\ifPDFTeX
  \usepackage[T1]{fontenc}
  \usepackage[utf8]{inputenc}
  \usepackage{textcomp} % provide euro and other symbols
\else % if luatex or xetex
  \usepackage{unicode-math} % this also loads fontspec
  \defaultfontfeatures{Scale=MatchLowercase}
  \defaultfontfeatures[\rmfamily]{Ligatures=TeX,Scale=1}
\fi
\usepackage{lmodern}
\ifPDFTeX\else
  % xetex/luatex font selection
  \setmainfont[]{Times New Roman}
\fi
% Use upquote if available, for straight quotes in verbatim environments
\IfFileExists{upquote.sty}{\usepackage{upquote}}{}
\IfFileExists{microtype.sty}{% use microtype if available
  \usepackage[]{microtype}
  \UseMicrotypeSet[protrusion]{basicmath} % disable protrusion for tt fonts
}{}
\makeatletter
\@ifundefined{KOMAClassName}{% if non-KOMA class
  \IfFileExists{parskip.sty}{%
    \usepackage{parskip}
  }{% else
    \setlength{\parindent}{0pt}
    \setlength{\parskip}{6pt plus 2pt minus 1pt}}
}{% if KOMA class
  \KOMAoptions{parskip=half}}
\makeatother
\usepackage{xcolor}
\usepackage[margin=1in]{geometry}
\usepackage{color}
\usepackage{fancyvrb}
\newcommand{\VerbBar}{|}
\newcommand{\VERB}{\Verb[commandchars=\\\{\}]}
\DefineVerbatimEnvironment{Highlighting}{Verbatim}{commandchars=\\\{\}}
% Add ',fontsize=\small' for more characters per line
\usepackage{framed}
\definecolor{shadecolor}{RGB}{248,248,248}
\newenvironment{Shaded}{\begin{snugshade}}{\end{snugshade}}
\newcommand{\AlertTok}[1]{\textcolor[rgb]{0.94,0.16,0.16}{#1}}
\newcommand{\AnnotationTok}[1]{\textcolor[rgb]{0.56,0.35,0.01}{\textbf{\textit{#1}}}}
\newcommand{\AttributeTok}[1]{\textcolor[rgb]{0.13,0.29,0.53}{#1}}
\newcommand{\BaseNTok}[1]{\textcolor[rgb]{0.00,0.00,0.81}{#1}}
\newcommand{\BuiltInTok}[1]{#1}
\newcommand{\CharTok}[1]{\textcolor[rgb]{0.31,0.60,0.02}{#1}}
\newcommand{\CommentTok}[1]{\textcolor[rgb]{0.56,0.35,0.01}{\textit{#1}}}
\newcommand{\CommentVarTok}[1]{\textcolor[rgb]{0.56,0.35,0.01}{\textbf{\textit{#1}}}}
\newcommand{\ConstantTok}[1]{\textcolor[rgb]{0.56,0.35,0.01}{#1}}
\newcommand{\ControlFlowTok}[1]{\textcolor[rgb]{0.13,0.29,0.53}{\textbf{#1}}}
\newcommand{\DataTypeTok}[1]{\textcolor[rgb]{0.13,0.29,0.53}{#1}}
\newcommand{\DecValTok}[1]{\textcolor[rgb]{0.00,0.00,0.81}{#1}}
\newcommand{\DocumentationTok}[1]{\textcolor[rgb]{0.56,0.35,0.01}{\textbf{\textit{#1}}}}
\newcommand{\ErrorTok}[1]{\textcolor[rgb]{0.64,0.00,0.00}{\textbf{#1}}}
\newcommand{\ExtensionTok}[1]{#1}
\newcommand{\FloatTok}[1]{\textcolor[rgb]{0.00,0.00,0.81}{#1}}
\newcommand{\FunctionTok}[1]{\textcolor[rgb]{0.13,0.29,0.53}{\textbf{#1}}}
\newcommand{\ImportTok}[1]{#1}
\newcommand{\InformationTok}[1]{\textcolor[rgb]{0.56,0.35,0.01}{\textbf{\textit{#1}}}}
\newcommand{\KeywordTok}[1]{\textcolor[rgb]{0.13,0.29,0.53}{\textbf{#1}}}
\newcommand{\NormalTok}[1]{#1}
\newcommand{\OperatorTok}[1]{\textcolor[rgb]{0.81,0.36,0.00}{\textbf{#1}}}
\newcommand{\OtherTok}[1]{\textcolor[rgb]{0.56,0.35,0.01}{#1}}
\newcommand{\PreprocessorTok}[1]{\textcolor[rgb]{0.56,0.35,0.01}{\textit{#1}}}
\newcommand{\RegionMarkerTok}[1]{#1}
\newcommand{\SpecialCharTok}[1]{\textcolor[rgb]{0.81,0.36,0.00}{\textbf{#1}}}
\newcommand{\SpecialStringTok}[1]{\textcolor[rgb]{0.31,0.60,0.02}{#1}}
\newcommand{\StringTok}[1]{\textcolor[rgb]{0.31,0.60,0.02}{#1}}
\newcommand{\VariableTok}[1]{\textcolor[rgb]{0.00,0.00,0.00}{#1}}
\newcommand{\VerbatimStringTok}[1]{\textcolor[rgb]{0.31,0.60,0.02}{#1}}
\newcommand{\WarningTok}[1]{\textcolor[rgb]{0.56,0.35,0.01}{\textbf{\textit{#1}}}}
\usepackage{longtable,booktabs,array}
\usepackage{calc} % for calculating minipage widths
% Correct order of tables after \paragraph or \subparagraph
\usepackage{etoolbox}
\makeatletter
\patchcmd\longtable{\par}{\if@noskipsec\mbox{}\fi\par}{}{}
\makeatother
% Allow footnotes in longtable head/foot
\IfFileExists{footnotehyper.sty}{\usepackage{footnotehyper}}{\usepackage{footnote}}
\makesavenoteenv{longtable}
\usepackage{graphicx}
\makeatletter
\def\maxwidth{\ifdim\Gin@nat@width>\linewidth\linewidth\else\Gin@nat@width\fi}
\def\maxheight{\ifdim\Gin@nat@height>\textheight\textheight\else\Gin@nat@height\fi}
\makeatother
% Scale images if necessary, so that they will not overflow the page
% margins by default, and it is still possible to overwrite the defaults
% using explicit options in \includegraphics[width, height, ...]{}
\setkeys{Gin}{width=\maxwidth,height=\maxheight,keepaspectratio}
% Set default figure placement to htbp
\makeatletter
\def\fps@figure{htbp}
\makeatother
\setlength{\emergencystretch}{3em} % prevent overfull lines
\providecommand{\tightlist}{%
  \setlength{\itemsep}{0pt}\setlength{\parskip}{0pt}}
\setcounter{secnumdepth}{-\maxdimen} % remove section numbering
\ifLuaTeX
\usepackage[bidi=basic]{babel}
\else
\usepackage[bidi=default]{babel}
\fi
\babelprovide[main,import]{spanish}
\ifPDFTeX
\else
\babelfont{rm}[]{Times New Roman}
\fi
% get rid of language-specific shorthands (see #6817):
\let\LanguageShortHands\languageshorthands
\def\languageshorthands#1{}
\ifLuaTeX
  \usepackage{selnolig}  % disable illegal ligatures
\fi
\usepackage{bookmark}
\IfFileExists{xurl.sty}{\usepackage{xurl}}{} % add URL line breaks if available
\urlstyle{same}
\hypersetup{
  pdflang={es-ES},
  hidelinks,
  pdfcreator={LaTeX via pandoc}}

\title{U9-ADMINISTRACIÓ i CONFIGURACIÓ AVANÇADA D'UBUNTU (at i cron)}
\usepackage{etoolbox}
\makeatletter
\providecommand{\subtitle}[1]{% add subtitle to \maketitle
  \apptocmd{\@title}{\par {\large #1 \par}}{}{}
}
\makeatother
\subtitle{~Programar tasques. \textbf{at i cron}}
\author{}
\date{\vspace{-2.5em}}

\begin{document}
\maketitle

\setstretch{1.5}
\section{\texorpdfstring{1. ÚS
D'\texttt{at}}{1. ÚS D'at}}\label{uxfas-dat}

L'ordre \texttt{at} permet programar tasques que s'executaran
\textbf{una sola vegada} en el futur.

\subsection{\texorpdfstring{1.1 Activació del servei
\texttt{atd}}{1.1 Activació del servei atd}}\label{activaciuxf3-del-servei-atd}

\begin{Shaded}
\begin{Highlighting}[]
\FunctionTok{sudo}\NormalTok{ systemctl start atd}
\FunctionTok{sudo}\NormalTok{ systemctl enable atd}
\end{Highlighting}
\end{Shaded}

\subsubsection{\texorpdfstring{Exemple 1: Crear una tasca amb
\texttt{at}}{Exemple 1: Crear una tasca amb at}}\label{exemple-1-crear-una-tasca-amb-at}

\begin{enumerate}
\def\labelenumi{\arabic{enumi}.}
\tightlist
\item
  Escriu una ordre que vols executar a una hora específica:
\end{enumerate}

\begin{Shaded}
\begin{Highlighting}[]
\BuiltInTok{echo} \StringTok{"echo \textquotesingle{}Hola, món!\textquotesingle{} \textgreater{}\textgreater{} \textasciitilde{}/hola.txt"} \KeywordTok{|} \ExtensionTok{at}\NormalTok{ 14:00}
\end{Highlighting}
\end{Shaded}

\begin{Shaded}
\begin{Highlighting}[]
\ExtensionTok{at}\NormalTok{ 14:00 }\AttributeTok{{-}f}\NormalTok{ backup.sh}
\end{Highlighting}
\end{Shaded}

\begin{Shaded}
\begin{Highlighting}[]
\FunctionTok{sudo}\NormalTok{ apt update }\KeywordTok{\&\&} \FunctionTok{sudo}\NormalTok{ apt upgrade }\AttributeTok{{-}y}
\end{Highlighting}
\end{Shaded}

Això escriurà ``Hola, món!'' al fitxer \texttt{hola.txt} a les
\textbf{14:00}.

\begin{enumerate}
\def\labelenumi{\arabic{enumi}.}
\setcounter{enumi}{1}
\tightlist
\item
  També pots utilitzar formats de temps relatius:
\end{enumerate}

\begin{Shaded}
\begin{Highlighting}[]
\BuiltInTok{echo} \StringTok{"sudo apt update"} \KeywordTok{|} \ExtensionTok{at}\NormalTok{ now + 2 minutes}
\end{Highlighting}
\end{Shaded}

Això executarà l'ordre \textbf{\texttt{sudo\ apt\ update}} d'ací a 2
minuts.

\subsection{1.2 Llistar tasques
programades}\label{llistar-tasques-programades}

\begin{Shaded}
\begin{Highlighting}[]
\ExtensionTok{atq}
\end{Highlighting}
\end{Shaded}

\subsection{1.3 Eliminar una tasca
programada}\label{eliminar-una-tasca-programada}

\begin{enumerate}
\def\labelenumi{\arabic{enumi}.}
\item
  Troba l'ID de la tasca amb \texttt{atq}.
\item
  Elimina-la amb:
\end{enumerate}

\begin{Shaded}
\begin{Highlighting}[]
\ExtensionTok{atrm} \OperatorTok{\textless{}}\NormalTok{id}\OperatorTok{\textgreater{}}
\end{Highlighting}
\end{Shaded}

\begin{center}\rule{0.5\linewidth}{0.5pt}\end{center}

\section{\texorpdfstring{2. ÚS DE CRON
\texttt{cron}}{2. ÚS DE CRON cron}}\label{uxfas-de-cron-cron}

\texttt{cron} és ideal per programar \textbf{tasques recurrents}.

\subsection{2.1 Configuració del cron}\label{configuraciuxf3-del-cron}

\begin{enumerate}
\def\labelenumi{\arabic{enumi}.}
\tightlist
\item
  Obriu l'editor del cron:
\end{enumerate}

\begin{Shaded}
\begin{Highlighting}[]
\FunctionTok{crontab} \AttributeTok{{-}e}
\end{Highlighting}
\end{Shaded}

\begin{enumerate}
\def\labelenumi{\arabic{enumi}.}
\setcounter{enumi}{1}
\tightlist
\item
  Afegiu una línia per programar una tasca. Per exemple:
\end{enumerate}

\begin{verbatim}
0 6 * * * echo "Bon dia!" >> ~/bon_dia.txt
\end{verbatim}

Això escriurà ``Bon dia!'' al fitxer \texttt{bon\_dia.txt} cada dia a
les \textbf{6:00 AM}.

\subsection{2.2 Sintaxi de cron}\label{sintaxi-de-cron}

\begin{itemize}
\item
  \textbf{Minut (0-59)}
\item
  \textbf{Hora (0-23)}
\item
  \textbf{Dia del mes (1-31)}
\item
  \textbf{Mes (1-12)}
\item
  \textbf{Dia de la setmana (0-7)} (on 0 i 7 són diumenge)
\end{itemize}

Exemples:

\begin{enumerate}
\def\labelenumi{\arabic{enumi}.}
\tightlist
\item
  Executar una tasca cada 5 minuts:
\end{enumerate}

\begin{verbatim}
*/5 * * * * /path/to/script.sh
\end{verbatim}

\begin{enumerate}
\def\labelenumi{\arabic{enumi}.}
\setcounter{enumi}{1}
\tightlist
\item
  Programar una tasca el primer dia de cada mes a les 12 del migdia:
\end{enumerate}

\begin{verbatim}
0 12 1 * * /path/to/script.sh
\end{verbatim}

\begin{enumerate}
\def\labelenumi{\arabic{enumi}.}
\setcounter{enumi}{2}
\tightlist
\item
  Reiniciar el sistema cada diumenge a mitjanit:
\end{enumerate}

\begin{verbatim}
0 0 * * 0 sudo reboot
\end{verbatim}

\subsection{2.3 Llistar tasques}\label{llistar-tasques}

Per veure les tasques configurades:

\begin{Shaded}
\begin{Highlighting}[]
\FunctionTok{crontab} \AttributeTok{{-}l}
\end{Highlighting}
\end{Shaded}

\subsection{2.4 Eliminar tasca}\label{eliminar-tasca}

Per eliminar totes les tasques del cron:

\begin{Shaded}
\begin{Highlighting}[]
\FunctionTok{crontab} \AttributeTok{{-}r}
\end{Highlighting}
\end{Shaded}

\begin{center}\rule{0.5\linewidth}{0.5pt}\end{center}

\section{\texorpdfstring{3. DIFERÈNCIES ENTRE \texttt{at} i
\texttt{cron}}{3. DIFERÈNCIES ENTRE at i cron}}\label{diferuxe8ncies-entre-at-i-cron}

\begin{longtable}[]{@{}ll@{}}
\toprule\noalign{}
\textbf{\texttt{at}} & \textbf{\texttt{cron}} \\
\midrule\noalign{}
\endhead
\bottomrule\noalign{}
\endlastfoot
Execució \textbf{una sola vegada} & Execució \textbf{recurrent} \\
Ordres simples i puntuals & Ideal per programacions periòdiques \\
Requereix el servei \texttt{atd} & Requereix el dimoni \texttt{cron} \\
\end{longtable}

\end{document}
