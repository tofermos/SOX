% Options for packages loaded elsewhere
\PassOptionsToPackage{unicode}{hyperref}
\PassOptionsToPackage{hyphens}{url}
%
\documentclass[
  a4paper,
]{article}
\usepackage{amsmath,amssymb}
\usepackage{setspace}
\usepackage{iftex}
\ifPDFTeX
  \usepackage[T1]{fontenc}
  \usepackage[utf8]{inputenc}
  \usepackage{textcomp} % provide euro and other symbols
\else % if luatex or xetex
  \usepackage{unicode-math} % this also loads fontspec
  \defaultfontfeatures{Scale=MatchLowercase}
  \defaultfontfeatures[\rmfamily]{Ligatures=TeX,Scale=1}
\fi
\usepackage{lmodern}
\ifPDFTeX\else
  % xetex/luatex font selection
\fi
% Use upquote if available, for straight quotes in verbatim environments
\IfFileExists{upquote.sty}{\usepackage{upquote}}{}
\IfFileExists{microtype.sty}{% use microtype if available
  \usepackage[]{microtype}
  \UseMicrotypeSet[protrusion]{basicmath} % disable protrusion for tt fonts
}{}
\makeatletter
\@ifundefined{KOMAClassName}{% if non-KOMA class
  \IfFileExists{parskip.sty}{%
    \usepackage{parskip}
  }{% else
    \setlength{\parindent}{0pt}
    \setlength{\parskip}{6pt plus 2pt minus 1pt}}
}{% if KOMA class
  \KOMAoptions{parskip=half}}
\makeatother
\usepackage{xcolor}
\usepackage[margin=1in]{geometry}
\usepackage{longtable,booktabs,array}
\usepackage{calc} % for calculating minipage widths
% Correct order of tables after \paragraph or \subparagraph
\usepackage{etoolbox}
\makeatletter
\patchcmd\longtable{\par}{\if@noskipsec\mbox{}\fi\par}{}{}
\makeatother
% Allow footnotes in longtable head/foot
\IfFileExists{footnotehyper.sty}{\usepackage{footnotehyper}}{\usepackage{footnote}}
\makesavenoteenv{longtable}
\usepackage{graphicx}
\makeatletter
\def\maxwidth{\ifdim\Gin@nat@width>\linewidth\linewidth\else\Gin@nat@width\fi}
\def\maxheight{\ifdim\Gin@nat@height>\textheight\textheight\else\Gin@nat@height\fi}
\makeatother
% Scale images if necessary, so that they will not overflow the page
% margins by default, and it is still possible to overwrite the defaults
% using explicit options in \includegraphics[width, height, ...]{}
\setkeys{Gin}{width=\maxwidth,height=\maxheight,keepaspectratio}
% Set default figure placement to htbp
\makeatletter
\def\fps@figure{htbp}
\makeatother
\setlength{\emergencystretch}{3em} % prevent overfull lines
\providecommand{\tightlist}{%
  \setlength{\itemsep}{0pt}\setlength{\parskip}{0pt}}
\setcounter{secnumdepth}{-\maxdimen} % remove section numbering
\ifLuaTeX
\usepackage[bidi=basic]{babel}
\else
\usepackage[bidi=default]{babel}
\fi
\babelprovide[main,import]{catalan}
% get rid of language-specific shorthands (see #6817):
\let\LanguageShortHands\languageshorthands
\def\languageshorthands#1{}
\ifLuaTeX
  \usepackage{selnolig}  % disable illegal ligatures
\fi
\usepackage{bookmark}
\IfFileExists{xurl.sty}{\usepackage{xurl}}{} % add URL line breaks if available
\urlstyle{same}
\hypersetup{
  pdftitle={U1.Introducció als SOX},
  pdfauthor={@tofermos 2024},
  pdflang={ca-ES},
  hidelinks,
  pdfcreator={LaTeX via pandoc}}

\title{U1.Introducció als SOX}
\usepackage{etoolbox}
\makeatletter
\providecommand{\subtitle}[1]{% add subtitle to \maketitle
  \apptocmd{\@title}{\par {\large #1 \par}}{}{}
}
\makeatother
\subtitle{Activitat Unitat 1}
\author{@tofermos 2024}
\date{}

\begin{document}
\maketitle

{
\setcounter{tocdepth}{2}
\tableofcontents
}
\setstretch{1.5}
\newpage
\renewcommand\tablename{Tabla}

\begin{quote}
NOTA PRÈVIA:

Abans de començar, estudia el tema i llig TOTA la pràctica. Hi pot haver
un apartat que no pugues contestar fins el final o una dada que te la
proporcionen després.
\end{quote}

\section{1 Sobre la xarxa en
VirtualBox}\label{sobre-la-xarxa-en-virtualbox}

\subsection{1.1 Modes de la xarxa}\label{modes-de-la-xarxa}

Després d'instal·lar un Windows 1x en le sdos màquines, comprova les
diferents configuracions de la xarxa del VisualBox i ompli la Taula
``Resultat''.

Per a tal fi has de:

\begin{itemize}
\tightlist
\item
  Provar totes les opcions i fes un ifconfig des del CMD per comprovar
  la IP.
\item
  Mirar també quina IP té el teu PC anfritrió (Ubuntu a l'aula o Windows
  a casa).
\item
  Fer un ping a la IP de l'amfitrió des de la MV.
\end{itemize}

\emph{Fes captures}

\begin{longtable}[]{@{}llll@{}}
\toprule\noalign{}
& NAT & XARXA INTERNA & PONT \\
\midrule\noalign{}
\endhead
\bottomrule\noalign{}
\endlastfoot
Té accés a internet & & & \\
Pot comunicar-se amb altres MV & & & \\
Es pot comunicar amb la màquina real & & & \\
Hem de configurar alguna cosa a Linux/Windows & & & \\
IP MV1 & & & \\
IP MV2 & & & \\
IP Amfitrió INTERNA & & & \\
Es comuniquen les MV(ping) & & & \\
Es comuniquen amb l'amfitrió & & & \\
\end{longtable}

\subsection{1.2 Xarxa local}\label{xarxa-local}

Anem a unir dos MV Windows 1x per fer un WorkGroup. Fes els canvis que
cregues.

\emph{Fes captures}

\section{2 Sobre la compartició en
Windows.}\label{sobre-la-comparticiuxf3-en-windows.}

Revisa que els dos Windows no tenen cap restricció del software del
sistema que els impedisca compartir algun recurs.

en acabar, comprova que les dos màquines tenen connexió.

\emph{Fes catptures}

\section{3 Botiga amb 2 PC}\label{botiga-amb-2-pc}

\subsection{3.1 Crea la següent
configuració}\label{crea-la-seguxfcent-configuraciuxf3}

\begin{longtable}[]{@{}llll@{}}
\toprule\noalign{}
Nom del PC & IP & Recurs compartit & Usuari local \\
\midrule\noalign{}
\endhead
\bottomrule\noalign{}
\endlastfoot
PC-Mostrador & 192.168.1.X /24 &
C:\textbackslash DADES\textbackslash VendesDia & elteunom \\
PC-Oficina & 192.168.1.(X+1)Y /24 & & elteucognom \\
\end{longtable}

\textbf{Indicacions}

\begin{itemize}
\item
  Usaràs el teu nom de pila per a un usuari ``elteunom'' i
  ``elteucognom'' per a l'altre.
\item
  La X de la IP serà la del teu lloc de classe.
\end{itemize}

\subsection{3.2 Compartició
controlada.}\label{comparticiuxf3-controlada.}

Assegura't que en la carpeta compartida del mostrador, a banda de Joan,
només por accedir la gerent (SraRoser) pero sense poder modificar, sol
llegir.

\subsection{3.3 Respon de forma raonada als següents
dubtes.}\label{respon-de-forma-raonada-als-seguxfcents-dubtes.}

\begin{enumerate}
\def\labelenumi{\alph{enumi}.}
\tightlist
\item
  Podríem compartir un altra carpeta en
  C:\textbackslash users\textbackslash elteunom\textbackslash{}``Mis
  Documentos''
\item
  Consideres que és un model de xarxa el Workgroup acceptable
\item
  Si hagueres de muntar-lo en la FCT, ¿què necessitaries a banda dels
  dos PC i una ISO del Windows 11?
\end{enumerate}

\end{document}
