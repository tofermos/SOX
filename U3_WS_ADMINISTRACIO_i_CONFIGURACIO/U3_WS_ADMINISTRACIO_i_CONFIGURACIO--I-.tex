% Options for packages loaded elsewhere
\PassOptionsToPackage{unicode}{hyperref}
\PassOptionsToPackage{hyphens}{url}
%
\documentclass[
  a4paper,
]{article}
\usepackage{amsmath,amssymb}
\usepackage{setspace}
\usepackage{iftex}
\ifPDFTeX
  \usepackage[T1]{fontenc}
  \usepackage[utf8]{inputenc}
  \usepackage{textcomp} % provide euro and other symbols
\else % if luatex or xetex
  \usepackage{unicode-math} % this also loads fontspec
  \defaultfontfeatures{Scale=MatchLowercase}
  \defaultfontfeatures[\rmfamily]{Ligatures=TeX,Scale=1}
\fi
\usepackage{lmodern}
\ifPDFTeX\else
  % xetex/luatex font selection
\fi
% Use upquote if available, for straight quotes in verbatim environments
\IfFileExists{upquote.sty}{\usepackage{upquote}}{}
\IfFileExists{microtype.sty}{% use microtype if available
  \usepackage[]{microtype}
  \UseMicrotypeSet[protrusion]{basicmath} % disable protrusion for tt fonts
}{}
\makeatletter
\@ifundefined{KOMAClassName}{% if non-KOMA class
  \IfFileExists{parskip.sty}{%
    \usepackage{parskip}
  }{% else
    \setlength{\parindent}{0pt}
    \setlength{\parskip}{6pt plus 2pt minus 1pt}}
}{% if KOMA class
  \KOMAoptions{parskip=half}}
\makeatother
\usepackage{xcolor}
\usepackage[margin=1in]{geometry}
\usepackage{graphicx}
\makeatletter
\def\maxwidth{\ifdim\Gin@nat@width>\linewidth\linewidth\else\Gin@nat@width\fi}
\def\maxheight{\ifdim\Gin@nat@height>\textheight\textheight\else\Gin@nat@height\fi}
\makeatother
% Scale images if necessary, so that they will not overflow the page
% margins by default, and it is still possible to overwrite the defaults
% using explicit options in \includegraphics[width, height, ...]{}
\setkeys{Gin}{width=\maxwidth,height=\maxheight,keepaspectratio}
% Set default figure placement to htbp
\makeatletter
\def\fps@figure{htbp}
\makeatother
\setlength{\emergencystretch}{3em} % prevent overfull lines
\providecommand{\tightlist}{%
  \setlength{\itemsep}{0pt}\setlength{\parskip}{0pt}}
\setcounter{secnumdepth}{-\maxdimen} % remove section numbering
\ifLuaTeX
\usepackage[bidi=basic]{babel}
\else
\usepackage[bidi=default]{babel}
\fi
\babelprovide[main,import]{catalan}
% get rid of language-specific shorthands (see #6817):
\let\LanguageShortHands\languageshorthands
\def\languageshorthands#1{}
\ifLuaTeX
  \usepackage{selnolig}  % disable illegal ligatures
\fi
\usepackage{bookmark}
\IfFileExists{xurl.sty}{\usepackage{xurl}}{} % add URL line breaks if available
\urlstyle{same}
\hypersetup{
  pdftitle={U3. WINDOWS SERVER. ADMINISTRACIÓ I CONFIGURACIÓ},
  pdfauthor={@tofermos 2024},
  pdflang={ca-ES},
  hidelinks,
  pdfcreator={LaTeX via pandoc}}

\title{U3. WINDOWS SERVER. ADMINISTRACIÓ I CONFIGURACIÓ}
\usepackage{etoolbox}
\makeatletter
\providecommand{\subtitle}[1]{% add subtitle to \maketitle
  \apptocmd{\@title}{\par {\large #1 \par}}{}{}
}
\makeatother
\subtitle{Introducció al servidor}
\author{@tofermos 2024}
\date{}

\begin{document}
\maketitle

{
\setcounter{tocdepth}{2}
\tableofcontents
}
\setstretch{1.5}
\newpage
\renewcommand\tablename{Tabla}

\section{1 Funcions d'un servidor}\label{funcions-dun-servidor}

Des del punt de vista del que series les funcions d'un Sistema Operatiu
de Xarxa trobem, com ja hem vist a l'anterior unitat, que es corresponen
a alguns dels Rols i Característiques. Podem dir que Windows Server les
implementa així. No tots els Rols i Característiques, menys encara, són
funcions principals.

\begin{enumerate}
\def\labelenumi{\arabic{enumi}.}
\item
  La funció de Servei de Directori que a Linux serà en OpenLDAP i vorem
  més avant, ací els el \textbf{Active Directory Domain Services (AD
  DS)}: Permet crear i gestionar dominis de forma centralitzada i
  còmoda.
\item
  La funció o servei de \textbf{DHCP Server}: Assigna automàticament
  adreces IP als dispositius de la xarxa.
\item
  \textbf{DNS Server}: Traduïx noms de domini a adreces IP, dins d'una
  xarxa o a internet. R \emph{Recordeu que quan configuràvem les IP en
  un WorkGroup NO indicàvem cap IP de servidor DNS. En un Domini, usem
  la ressolució de noms de DNS, molt més eficient}
\item
  \textbf{File and Storage Services}: Gestiona el sistema
  d'emmagatzematge de fitxers i carpetes compartides, quotes
  d'emmagatzematge, duplicació de dades\ldots{}
\item
  \textbf{Servici de backup de Seguretat de Windows Server}.
\item
  La connexió remota pot considerar-se com una funció dels servidors. En
  WS tenim \textbf{Remote Desktop Services (RDS)} que permet als usuaris
  la onnexió remota a escriptoris virtuals o aplicacions.
\item
  \textbf{Print and Document Services}: Permet gestionar impressores i
  compartir-les en la xarxa.
\item
  \textbf{Web Server (Internet Information Services (IIS)}. Servidor de
  webs.
\end{enumerate}

\section{2 Administració i configuració
bàsica}\label{administraciuxf3-i-configuraciuxf3-buxe0sica}

\subsection{Consoles i altres utilitats comuns a tots el sistemes
Windows}\label{consoles-i-altres-utilitats-comuns-a-tots-el-sistemes-windows}

Al curs de Windows 11 d'aquest repositori podreu trobar una guia més que
suficient sobre les utitlitas gràfiques del sistema Windows per
configurar i administrar una màquina.

\href{https://tofermos.github.io/Windows11/gestiodelequip/gestiodelequip.html}{Consoles
i altres utilitats}

\subsection{Consoles i altres utilitats específiques de Windows
Server}\label{consoles-i-altres-utilitats-especuxedfiques-de-windows-server}

A banda de les vistes en l'apartat anterior i que són comunes, la
pràctica totalitat, a tots els Windows tenim que, específicament de
Windows Server les consoles i utilitats següents:

\textbf{servermanager.exe} - Administrador de Servidors. Aquesta és la
utilitat (no es consola estrictament parlant) central per a gestionar el
servidor. Permet configurar rols i característiques, gestionar discos,
supervisar el rendiment, entre altres funcions.

\textbf{dcpromo.msc} - Promoció de controlador de domini: Utilitzada per
configurar un controlador de domini (AD DS), una funció exclusiva de
Windows Server.

\textbf{dnsmgmt.msc} - Gestió de DNS: Disponible en Windows Server per
gestionar zones i registres DNS.

\textbf{dhcpmgmt.msc} - Gestió de DHCP: Permet administrar el rol de
servidor DHCP per assignar adreces IP automàticament a dispositius de la
xarxa.

\textbf{fsmgmt.msc} - Carpetes compartides: Una consola específica per
gestionar carpetes i recursos compartits al servidor, encara que també
es pot trobar en versions professionals de Windows 10/11.

\emph{(els 2 següents no anem a mirar-los en SOX)}

\textbf{tsadmin.msc} o Remote Desktop Services Manager: Utilitzada per
gestionar sessions d'escriptori remot, més comuna en Windows Server per
administrar entorns d'escriptori remot (RDS).

\textbf{cluadmin.msc} - Gestió de Clúster de Failover: Disponible en
Windows Server per administrar clústers de tolerància a fallades i alta
disponibilitat, especialment útil per entorns crítics empresarials.

\subsection{PowerShell (El vorem més
avant)}\label{powershell-el-vorem-muxe9s-avant}

Més avant, si farem una ullada interessant al lleguatge d'scripts basat
en cmdLets (comandaments de Windows) molt avaçat i potent.

Si voleu consultar, teniu un curs de PowerShell en aquest repositori:

\href{https://github.com/tofermos/PowerShell}{Curs PowerShell}

\section{3 Administració i configuració de comptes
locals}\label{administraciuxf3-i-configuraciuxf3-de-comptes-locals}

Els comptes locals perden importància en un Domini. No obstant podeu
consultar el curs de Windows 1x d'aquest repositori, ja que és un tema
comú a tots els Windows Server.

\href{https://tofermos.github.io/Windows11/gestions/comptesLocals.html}{Comptes
locals}

\end{document}
