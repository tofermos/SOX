% Options for packages loaded elsewhere
\PassOptionsToPackage{unicode}{hyperref}
\PassOptionsToPackage{hyphens}{url}
%
\documentclass[
  a4paper,
]{article}
\usepackage{amsmath,amssymb}
\usepackage{setspace}
\usepackage{iftex}
\ifPDFTeX
  \usepackage[T1]{fontenc}
  \usepackage[utf8]{inputenc}
  \usepackage{textcomp} % provide euro and other symbols
\else % if luatex or xetex
  \usepackage{unicode-math} % this also loads fontspec
  \defaultfontfeatures{Scale=MatchLowercase}
  \defaultfontfeatures[\rmfamily]{Ligatures=TeX,Scale=1}
\fi
\usepackage{lmodern}
\ifPDFTeX\else
  % xetex/luatex font selection
\fi
% Use upquote if available, for straight quotes in verbatim environments
\IfFileExists{upquote.sty}{\usepackage{upquote}}{}
\IfFileExists{microtype.sty}{% use microtype if available
  \usepackage[]{microtype}
  \UseMicrotypeSet[protrusion]{basicmath} % disable protrusion for tt fonts
}{}
\makeatletter
\@ifundefined{KOMAClassName}{% if non-KOMA class
  \IfFileExists{parskip.sty}{%
    \usepackage{parskip}
  }{% else
    \setlength{\parindent}{0pt}
    \setlength{\parskip}{6pt plus 2pt minus 1pt}}
}{% if KOMA class
  \KOMAoptions{parskip=half}}
\makeatother
\usepackage{xcolor}
\usepackage[margin=1in]{geometry}
\usepackage{graphicx}
\makeatletter
\def\maxwidth{\ifdim\Gin@nat@width>\linewidth\linewidth\else\Gin@nat@width\fi}
\def\maxheight{\ifdim\Gin@nat@height>\textheight\textheight\else\Gin@nat@height\fi}
\makeatother
% Scale images if necessary, so that they will not overflow the page
% margins by default, and it is still possible to overwrite the defaults
% using explicit options in \includegraphics[width, height, ...]{}
\setkeys{Gin}{width=\maxwidth,height=\maxheight,keepaspectratio}
% Set default figure placement to htbp
\makeatletter
\def\fps@figure{htbp}
\makeatother
\setlength{\emergencystretch}{3em} % prevent overfull lines
\providecommand{\tightlist}{%
  \setlength{\itemsep}{0pt}\setlength{\parskip}{0pt}}
\setcounter{secnumdepth}{-\maxdimen} % remove section numbering
\ifLuaTeX
\usepackage[bidi=basic]{babel}
\else
\usepackage[bidi=default]{babel}
\fi
\babelprovide[main,import]{catalan}
% get rid of language-specific shorthands (see #6817):
\let\LanguageShortHands\languageshorthands
\def\languageshorthands#1{}
\ifLuaTeX
  \usepackage{selnolig}  % disable illegal ligatures
\fi
\usepackage{bookmark}
\IfFileExists{xurl.sty}{\usepackage{xurl}}{} % add URL line breaks if available
\urlstyle{same}
\hypersetup{
  pdftitle={U3.Administració i configuració del Windows Server},
  pdfauthor={@tofermos 2024},
  pdflang={ca-ES},
  hidelinks,
  pdfcreator={LaTeX via pandoc}}

\title{U3.Administració i configuració del Windows Server}
\usepackage{etoolbox}
\makeatletter
\providecommand{\subtitle}[1]{% add subtitle to \maketitle
  \apptocmd{\@title}{\par {\large #1 \par}}{}{}
}
\makeatother
\subtitle{Activitat Unitat 3. Creació de domini}
\author{@tofermos 2024}
\date{}

\begin{document}
\maketitle

{
\setcounter{tocdepth}{2}
\tableofcontents
}
\setstretch{1.5}
\newpage
\renewcommand\tablename{Tabla}

\begin{quote}
NOTA PRÈVIA:

Abans de començar, estudieu la Unitat 3 i llegiu TOTA la pràctica. Hi
pot haver un apartat que no pugues contestar fins el final o una dada
que te la proporcionen després.
\end{quote}

\section{Descripció}\label{descripciuxf3}

En aquesta activitat anem a crear el domini. Abandonarem el model de
WorkGroup i passarem a una arquitectura clarament ja de Client/Servidor.

\begin{quote}
NOTA

On diu \emph{nomteu} has de posar el teu cognom primer o nom de pila
\end{quote}

\subsection{Pasos per crear el Domini}\label{pasos-per-crear-el-domini}

Seguiu la documentació del repositori (U3 Administració i configuració
II):

\begin{itemize}
\tightlist
\item
  Instal·lar el AD-DS
\item
  Promocionar el Windows Server a DC
\item
  Crear un nou bosc amb el nom de domini rel: \emph{nomteu.local} *Ens
  assegurem que el servidor serà el Servidor DNS.
\end{itemize}

\subsection{Pasos per crear algun objetes del
AD}\label{pasos-per-crear-algun-objetes-del-ad}

\begin{itemize}
\tightlist
\item
  Crea una UO amb el nom ``usuari\_DelegacioNord''
\item
  Crea un grup d'usuaris de la delegació creada amb el nom
  ``gr\_programadors''
\item
  Crea un usuari pertanyent al grup amb el nom: ``nomteu\_programador''
\item
  Assigna-li una contrassenya
\end{itemize}

\section{Afig el Windows 1x al
domini}\label{afig-el-windows-1x-al-domini}

Afig el PC al domini. Inicia sessió amb l'usuari del domini creat

1 Accedeix a: \emph{F:\textbackslash DADESSERVIDOR\textbackslash f1.txt}
compartida del servidor i obri'l des del client.

2 Deixa'l obert en Windows 1x. En l'apartat ``Sesiones'' de
\emph{fsmgmnt.exe}. Ara, veus alguna sessió del Windows 1x?

3 Força el tacament del fitxer obert en ``Ficheros abiertos''

4 Comprova que s'ha tancat.

\end{document}
